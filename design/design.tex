\chapter{Проектирование системы распознавания движений}


\section{Структура системы распознавания}

Система распознавания состоит из нескольких блоков (рис. \ref{fig:system}):

\pastepic{Структура системы распознавания}{design/system.png}{system}

\begin{enumerate}
\item На умных часах происходит сбор данных с сенсоров и передача их в "сыром" виде на смартфон через установленное Bluetooth-соединение
\item В режиме обучения данные со смартфона отправляются на облачное хранилище, откуда далее попадают на стационарный компьютер
\item Обучение классификаторов происходит в офлайн-режиме
\item Модель с оптимальными параметрами загружается в приложение на смартфоне
\item В рабочем режиме данные подаются на вход обученному классификатору, и результат обработки показывается пользователю
\end{enumerate}


\section{Формальная постановка задачи}

В наиболее абстрактном виде задача описывается следующим образом. 

Временные ряды представлены конечными последовательностями значений: $S=s_1,s_2...s_n$, где $s_i$ - вектор значений $\left\{x_i,y_i,z_i\right\}$. Дан конечный набор классов $C=c_1,c_2...c_m$. Известно, что каждый ряд принадлежит одному и только одному классу. 

Для некоторого множества $L$ рядов известна принадлежность ряда к классу - такое множество назовём \defn{обучающей выборкой}. Также имеется множество $T$ рядов, принадлежность которых к классам неизвестна - это \defn{тестовая выборка}.

Необходимо по данной обучающей выборке классифицировать элементы тестовой выборки с максимальной точностью.

\section{Алгоритмы выделения признаков}

\defn{Признаком} называется результат измерения некоторой характеристики объекта. Формально  — это отображение $f: X\to D_f $, где D\_f — множество допустимых значений признака. Вектор $\bigl( f_1(x),\ldots,f_n(x) \bigr)$ называется \defn{признаковым описанием} объекта $x \in X$ \cite{features_def}. Матрица, состоящая из строк - признаковых описаний объектов, является стандартным видом представления исходных данных во многих алгоритмах машинного обучения. 

Несмотря на то, что временные ряды можно непосредственно рассматривать как наборы признаков, часто имеет смысл выделять другие признаки (выполнять т.н. \defn{feature extraction}) по разным причинам: несовпадение длин рядов, наличие шума и т.д.

\subsection{Классификация на основе коэффициентов авторегрессии}

Модели авторегрессии широко используются в статистике для анализа и прогнозирования стационарных рядов. В их основе лежит представление временного ряда как выхода линейного фильтра, на вход которого подана последовательность нормально распределённых случайных величин:

$z_t=\mu+a_t+\phi_1a_{t-1}+\phi_2a_{t-2}...=\mu+\Phi(B)a_t$, 

где $z_t$ - элемент ряда с индексом $t$, $\mu$ - среднее значение (уровень) ряда, $a_i$ - случайная величина, действующая в момент времени $i$, $\Psi(B)=1+\psi B+\psi_2 B^2+...$ - передаточная функция фильтра, $B$ - лаговый оператор: $Bz_t=z_{t-1}$.

На практике используются модели, представляющие ряд в виде регрессии от конечного числа величин\cite{bj_ts}: 

\begin{itemize}
\item модель авторегресии (\defn{AR-модель}): 

$z_t=\mu+\sum_{i=1}^p \phi_iB^i z_t + a_t$. В данном случае очередное значение ряда линейно зависит от предыдущих $p$ значений ряда
\item модель скользящего среднего (\defn{MA-модель}):

$z_t=\mu + a_t + \sum_{i=1}^q -\theta_iB^i a_t $. В данном случае очередное значение ряда линейно зависит от $q$ предыдущих случайных величин
\item модель авторегрессии-случайного среднего (\defn{ARMA-модель}) представляет собой комбинацию указанных выше методов:

$z_t=\mu + a_t + \sum_{i=1}^p \phi_iB^i z_t + \sum_{i=1}^q -\theta_iB^i a_t $. Применение этой модели позволяет снизить суммарное число параметров, что упрощает дальнейший анализ
\end{itemize}

В качестве признаков объекта могут использоваться параметры $\phi_i, \theta_i, \mu$, дисперсия случайной величины $a_i$ - $\sigma^2$, а также порядки моделей.

\missing{перенести этот блок в общую часть?}
Обратим внимание на то, что в решаемой задаче временные ряды представляют собой последовательности векторов. Соответственно, необходимо каким-либо образом учитывать все элементы этих векторов:

\begin{itemize}
\item Наиболее простой вариант - предположить, что временные ряды, состоящие из отдельных элементов векторов (x, y, z), представляют собой независимые случайные процессы. В этом случае эти ряды можно анализировать по отдельности, а общая модель будет описываться конкатенацией параметров полученных подмоделей\missing{перефразировать}.
\item Учесть зависимости между рядами отдельных элементов. Для методов авторегрессии это, в частности, означает, что модель теперь представляется следующим образом (на примере AR-модели):

$U_t=M+\sum_{i=1}^p \Phi_iB^i U_t + A_t$,  где:

 $U_t=\colthree{x_t}{y_t}{z_t}$, $Phi_i=\mxthree{\phi_{xx}}{\phi_{yx}}{\phi_{zx}}{\phi_{xy}}{\phi_{yy}}{\phi_{zy}}{\phi_{xz}}{\phi_{yz}}{\phi_{zz}}$, 
 $A_t=\colthree{a_x}{a_y}{a_z}$

Модели, реализующие это, называются VAR(VMA,VARMA)-моделями. 
\end{itemize}


\missing{пруфы}
\missing{реализовать честный varma}


\subsection{Классификация на основе параметров скрытой марковской модели}

Дискретная марковская модель описывает систему, которая может в каждый данный дискретный момент времени находиться в одном из состояний $V=\{v_1...v_M\}$. Переход из одного состояния в другое зависит только от текущего состояния; таким образом, модель описывается матрицей $A: a_{ij}=P(v_i \to v_j)$. Это - наблюдаемая марковская модель: состояние системы является наблюдаемой величиной. 

Более общим и практичным подходом является аппарат \defn{скрытых марковских моделей} (СММ) - обобщение описанной концепции. В нём наблюдаемые события $v_k$ являются некоторой вероятностной функцией текущего состояния $s_j$, что описывается матрицей $B: b_j(k)=P[v_k | s_j]$. Кроме того, задаётся распределение вероятностей начального состояния $\Pi: \pi_i=P[q_1=s_i]$. 

Можно использовать СММ для моделирования временных рядов. Применительно к решаемой задаче наблюдаемые события $v_k$ - векторы-элементы временного ряда. Если удастся задать параметры модели так, чтобы последовательность наблюдаемых событий соответствовала данному временному ряду, то эти параметры можно будет использовать в качестве признаков для дальнейшей классификации.

Для нахождения оптимальных (максимизирующих вероятность наблюдения требуемой последовательности) параметров СММ используется \defn{алгоритм Баума-Велша}. Пусть $O_1..O_t$ - последовательность наблюдаемых состояний, а $Q_1..Q_t$ - последовательность скрытых состояний СММ. Вводятся следующие выражения:
\begin{itemize}
\item $\alpha_t(i)=P(O_1, ... O_t , Q_t=i)$ - прямая переменная
\item $\beta_t(i)=P(O_{t+1}, ... O_T , Q_t=i)$ - обратная переменная
\end{itemize}
Для их вычисления используются процедуры прямого и обратного хода соответственно \cite{hmm_review}. Они показывают для выбранных момента времени и скрытого состояния, какова вероятность появления наблюдаемой последовательности до и после момента $t$ соответственно. 

На их основе вводятся также:
\begin{itemize}
\item $\gamma_t(i)=P(q_t=i | O)=\dfrac{\alpha_t(i)\beta_t(i)}{\sum_k^N\alpha_t(k)\beta_t(k)}$ - вероятность нахождения в $i$ состоянии в момент времени $t$. Если эту величину просуммировать по всем $t$, то результат можно рассмотреть как ожидаемое количество переходов из состояния $i$
\item $\varepsilon_t(i,j)=P(q_t=i, q_{t+1}=j | O)=\dfrac{\alpha_t(i)a_{ij}b_j(O_{t+1})\beta_{t+1}(i)}{\sum_u^N\sum_v^N\alpha_t(u)a_{uv}b_v(O_{t+1})\beta_{t+1}(v)}$ - вероятность перехода из $i$ в $j$ в момент времени $t$. Если эту величину просуммировать по всем $t$, то результат можно рассмотреть как ожидаемое количество переходов из состояния $i$ в $j$
\end{itemize}

На основе этих формул можно произвести переоценку параметров $\pi, A, B$ и рассмотреть их смысл с точки зрения частоты переходов между состояниями:

\begin{itemize}
\item $\overline{\pi_i}=\gamma_1(i)$ - ожидаемое число переходов из $i$ в момент $t=1$
\item $\overline{a_{ij}}=\dfrac{\sum_t^T\varepsilon_t(i,j)}{\sum_t^T\gamma_t(i)}$ - ожидаемое число переходов из $i$ в $j$
\item $\overline{b_{j}(k)}=\dfrac{\sum_t^T\gamma_t(i)*b_j(k)}{\sum_t^T\gamma_t(i)}$ - ожидаемое число переходов из $i$ при наблюдении $k$
\end{itemize}

Баумом и его коллегами было показано, что этот алгоритм сходится к локальному оптимуму значений параметров \cite{hmm_conv_proof}.


\subsection{Выделение признаков частотной области}

Рассмотренные выше методы пользовались представлением сигнала как последовательности векторов. Между тем, большую информацию может предоставить спектр сигнала. Для его получения воспользуемся дискретным преобразованием Фурье (\defn{ДПФ}):

$S_k=\sum_{i=0}^{N-1}s_i*exp(-j\dfrac{2\pi}{N}ik), k=0..N-1$, где $S_k$ - комплексные коэффициенты Фурье, $s_i$ - отсчёты сигнала.

\missing{нужно ли приводить алгоритм ДПФ?}

Видно, что сложность расчёта спектра - $O(N^2)$ (суммирование $N$ точек сигнала для каждой из $N$ точек спектра). Для ускорения расчётов применяют подвид ДПФ - быстрое преобразование Фурье (\defn{БПФ}). Оно основано на том, что выполнение БПФ над $2M$ точками можно свести к двум независимым БПФ размерности $M$ и их объединению. 

Приведём один из таких алгоритмов - прореживание по времени \cite{fft_alg}. Пусть $y_i=s_{2i}$ и $z_i=s_{2i+1}$ - последовательности из чётных и нечётных элементов исходного ряда соответственно. Тогда

$S_k=\sum_{i=0}^{N/2-1}y_i*exp(-j\dfrac{2\pi}{N}2ik)+\sum_{i=0}^{N/2-1}z_i*exp(-j\dfrac{2\pi}{N}(2i+1)k)=\sum_{i=0}^{N/2-1}y_i*exp(-j\dfrac{2\pi}{N/2}2k)+exp(-j\dfrac{2\pi}{N}k)\sum_{i=0}^{N/2-1}z_i*exp(-j\dfrac{2\pi}{N/2}ik)$.

Таким образом, получили два ДПФ размерности $N/2$. Искомое ДПФ выражается как 
$S_k=Y_k+exp(-j\dfrac{2\pi}{N}k)Z_k$. Так как ДПФ размерности $N/2$ даёт $N/2$ коэффициентов, воспользуемся свойством периодичности спектра: $Y_{i+N/2}=Y_i, Z_{i+N/2}=Z_i$; следовательно, коэффициенты $S_{N/2}..S_{N}$ будут равны соответственно $S_{1}..S_{N/2-1}$. 

Схему, выполняющую соответствующие вычисления, можно представить следующим образом:

\pastepic{Вычисление N-точечного ДПФ через два N/2-точечных}{algorithms/fft.png}{fft}

На каждое половинное ДПФ выполняется $N^2/4$ операций; кроме того, производится $N/2$ умножений на комплексную экспоненту. В итоге такой алгоритм требует $2*N^2/4+N/2=N(N+1)/2$ операций - практически вдвое меньше, чем обычное ДПФ. Если $N$ является степенью двойки, то такое деление можно выполнять на каждом шаге и, таким образом, свести число операций к $N*log_2(N)$, что при больших $N$ даёт существенный прирост производительности.

В качестве выделяемых признаков можно использовать непосредственно спектральные коэффициенты или другие характеристики частотной области, которые можно из них получить.

Применение ДПФ в чистом виде может не дать достаточно показательные признаки. Дело в том, что, зная только состав частот в сигнале, мы теряем информацию о том, в какие моменты времени эти частоты встречались сигналы. Пример такого сигнала и его спектра представлен на \ref{fig:different_freqs}.

\pastepic{Сигнал, состоящий из нескольких отрезков с разными частотами на каждой}{design/different_freqs.png}{different_freqs}

В данном сигнале есть четыре частотные компоненты, каждая из которых действует на протяжении короткого отрезка времени. Зная только его спектр, невозможно сказать, в какой момент времени какая частота была активна в сигнале. Следовательно, для корректного анализа таких сигналов необходимо учитывать как спектр, так и момент времени, для которого производится расчёт. 

Такое преобразование называется \defn{оконным преобразованием Фурье (Short-time Fourier transform)} и представляется следующим образом:

$S_k(m)=\sum_{i=0}^{N-1} s_i*W_{i-m}-*exp(-j\dfrac{2\pi}{N}ik), k=0..N-1$, где $W_{i-m}$ - коэффициенты оконной функции, а $m$ - её сдвиг относительно сигнала. С её помощью происходит выделение конкретных участков сигнала, и ДПФ производится уже над сигналом $s_i*W_{i-m}$. Таким образом, спектр сигнала становится функцией от времени (переменной $m$). Это позволяет анализировать сигналы, подобные приведённому на рис. \ref{fig:different_freqs}\cite{wavelet_tutorial}.

При применении оконного преобразования мы сталкиваемся со следующей проблемой. При увеличении ширины окна мы теряем информацию о временной составляющей (при максимально возможном окне получаем обычное БПФ без информации о времени). При снижении ширины окна мы теряем информацию о спектральной составляющей: разрешающая способность ДПФ равна $df=F_s/N$, где $F_s$ - частота дискретизации, а $N$ - количество точек; так как $df$ обратно пропорционально $N$, при сужении окна разрешающая способность ухудшается. Поэтому необходимо соблюдать баланс между этими видами информации. 

\pastepic{Сигнал с кратковременным высокочастотным пиком}{design/peak_signal.png}{peak_signal}

При этом многие сигналы на практике имеют долговременные низкочастотные составляющие и кратковременные высокочастотные (см. рис \ref{fig:peak_signal}\cite{wavelet_tutorial}). В этом случае наиболее эффективно было бы улучшить разрешающую способность по частоте для отрезков с низкой частотой, а на участках с высокочастотными составляющими повысить разрешение по времени. 

Именно для этого был разработан алгоритм \defn{вейвлет-преобразования}. Его алгоритм можно описать следующим образом:

\begin{enumerate}
\item Исходный сигнал пропускается через два параллельных фильтра - низкочастотный и высокочастотный. Полученные сигналы имеют частотные диапазоны $0..f_{max}/2$ и $f_{max}/2..f_{max}$.
\item Каждый второй отсчёт с выходов фильтров можно отбросить, т.к. частотный диапазон каждого из них составляет половину исходного, и по теореме Котельникова это не приведёт к потере информации. При этом повышается разрешение по частоте (т.к. диапазон частот уменьшается) и снижается разрешение по времени (т.к. отсчёты теперь расположены реже)

\missing{уточнить, почему можно отбросить 1/2 высокочастотного выхода}
\item Сигнал с выхода низкочастотного фильтра подвергается шагам 1 и 2, в результате чего получается ещё два сигнала, которые соответствуют частотным диапазонам $0..f_{max}/4$ и $f_{max}/4..f_{max}/2$. При этом разрешение по частоте продолжает возрастать за счёт снижения разрешения по времени
\item Обработка сигнала фильтрами продолжается до тех пор, пока не будет получено требуемое разрешение по частоте
\end{enumerate}

Схему преобразования представлена на рис.\ref{fig:wavelet_scheme}\cite{wavelet_structure}:

\pastepic{Схема дискретного вейвлет-преобразования}{design/wavelet_scheme.png}{wavelet_scheme}




\section{Алгоритмы классификации}


\subsection{Нейронные сети}

% Нейронная сеть представляют собой множество связанных между собой элементарных вычислителей - нейронов, . Их можно 

Аппарат искусственных нейронных сетей позволяет представить решение некоторой задачи как совокупность связей между элементарными вычислителями - нейронами. Эта идея напоминает суть организации мозга - знания в нём также представлены синаптическими связями между нейронами. 

Отдельный нейрон представляет собой единицу обработки информации в сети. Он выполняет взвешенное суммирование входных сигналов, и полученная величина обрабатывается некоторой функцией активации (см. рис. \ref{fig:neuron}): 

$y_k=\varphi(\sum_{j=1}^m w_{kj}*x_j)$

\pastepic{Структура нейрона}{design/neuron.png}{neuron}

\missing{ссылка на хайкина}

Если в качестве функции активации взять, например, функцию Хевисайда 
\[
  \varphi(u)=\begin{cases}
               1, u>=0\\
               0, u<0 \\
              \end{cases}
\], то при правильной настройке весов нейрон может моделировать такие логические функции, как AND, OR, NOT (рис. \ref{fig:neuron_logic}):

%https://ru.wikipedia.org/wiki/%D0%98%D1%81%D0%BA%D1%83%D1%81%D1%81%D1%82%D0%B2%D0%B5%D0%BD%D0%BD%D1%8B%D0%B9_%D0%BD%D0%B5%D0%B9%D1%80%D0%BE%D0%BD
\pastepicthree{Моделирование логических функций AND, OR, NOT с помощью нейрона}{design/neuron_and.png}{design/neuron_or.png}{design/neuron_not.png}{neuron_logic}

\missing{ссылка на википедию}

\missing{поправить вёрстку пикчи}

Однако функция сложения по модулю 2 нейрон промоделировать уже не может. Это связано с тем, что нейрон фактически проводит в N-мерном пространстве гиперплоскость, разделяющую объекты двух классов. Для функции XOR такое разделение невозможно. Тем не менее, эта задача может быть решена композицией нейронов (рис. \ref{fig:neuron_xor}). Это свойство - повышение вычислительной мощности с усложнением структуры сети - позволяет создавать сети, моделирующие достаточно сложные системы. 

\pastepic{Решение задачи XOR нейронной сетью}{design/neuron_xor.png}{neuron_xor}


Для задач классификации наиболее распространённой является архитектура многослойного персептрона\missing{ссылка на хайкина}:

\begin{itemize}
\item нейроны имеют нелинейную дифференцируемую функцию активации
\item в сети есть один или более скрытых слоёв (не являющихся входом или выходом сети)
\item сеть является полносвязной - каждый нейрон слоя связан со всеми нейронами следующего слоя
\end{itemize}

Для решения задачи мультиклассового распознавания выходной слой сети реализуют с использованием функции softmax: $\varphi_i(u)=\dfrac{exp(u_i)}{\sum_{j=1}^N exp(u_j)}$, где $N$ - количество нейронов выходного слоя, равное числу классов. Таким образом, сумма значений, получаемых с выхода сети, равна единице, и совокупность этих значений можно рассматривать как вероятности принадлежности входного вектора тому или иному классу.

Наиболее простым алгоритмом обучения (подбора значений весов) нейронной сети является алгоритм обратного распространения ошибки.

\missing{нужно ли приводить алгоритм?}



\subsection{Наивный байесовский классификатор}

В основе данного метода лежит формула Байеса:

$P(Y_i|X)=\dfrac{P(X|Y)P(Y)}{P(X)}$

Здесь $P(X|Y_i)$ - вероятность найти элемент $X$ в классе $Y_i$, $P(Y_i)$ - вероятность найти какой-то элемент класса $Y_i$ среди всего множества элементов, $P(X)$ - вероятность найти элемент $X$ среди всего множества элементов, $P(Y_i|X)$ - вероятность того, что данный элемент $X$ принадлежит классу $Y_i$.

Задача классификации в этом случае сводится к выбору класса $Y_{opt}$, который максимизирует вероятность принадлежности к этому классу: $Y_{opt}=argmax_{Y_i} P(Y_i|X)=argmax_{Y_i} \dfrac{P(X|Y)P(Y)}{P(X)}$.

Так как $P(X)$ является для данного элемента константой, в расчётах её можно не учитывать. $P(Y_i)$ можно рассчитать как отношение числа элементов класса $Y_i$ к общему числу элементов выборки. 

Для расчёта $P(X|Y_i)$ необходимо представить $X$ как набор признаков: $P(X|Y_i)=P(c_1, c_2 .. c_m | Y_i)$, где набор $c_j$ - множество всех признаков объекта. В общем случае эта вероятность рассчитывается с учётом зависимостей между признаками: $P(X|Y_i)=P(c_1|Y_i)*P(c_2|c_1,Y_i)*..*P(c_m|c_1,..c_{m-1},Y_i)$. Восстановление зависимостей между признаками по обучающей выборке - достаточно трудная задача.

Метод наивной байесовской классификации предполагает, что признаки являются независимыми: $P(c_2|c_1,Y_i)=P(c_2|Y_i)$ и т.д. Тогда для расчёта $P(X|Y_i)$ необходимо рассчитать $m$ вероятностей того, что $j$ признак примет значение $c_j$ при условии принадлежности элемента к классу $Y_i$. На практике это предположение справедливо далеко не всегда, однако эксперименты показали, что это упрощение позволяет эффективно решать многие задачи.

В решаемой задаче $c_j$ можно рассматривать как непрерывные величины. Один из способов рассчитать $P(c_j|Y_i)$ - использовать вероятность по Гауссу:

$P(c_j|Y_i)=\dfrac{1}{\sqrt{2\pi\sigma_i^2}}*exp(-\dfrac{(c_j-\mu_i)^2}{2\sigma_i^2})$.

В процессе обучения параметры распределения $\mu_i$ и $\sigma_i^2$ рассчитываются по значениям $c_j$ из обучающей выборки. 

\missing{ссылки}

% http://i.stanford.edu/pub/cstr/reports/cs/tr/79/773/CS-TR-79-773.pdf
% http://bazhenov.me/blog/2012/06/11/naive-bayes.html
% http://scikit-learn.org/stable/modules/naive_bayes.html#gaussian-naive-bayes
% http://i.stanford.edu/pub/cstr/reports/cs/tr/79/773/CS-TR-79-773.pdf - пока что не использовано


\subsection{Применение линейного дискриминантного анализа как классификатора}

Метод \defn{линейного дискриминантного анализа (LDA)} применяется для того, чтобы снизить размерность исходных данных. Это осуществляется путём проекции на прямую, описываемую вектором $w$ в $N$-мерном пространстве.

Прямая выбирается так, чтобы проекции элементов одного класса располагались близко друг к другу, а проекции элементов других классов были отдалены (см. пример проекции на рис.\ref{fig:lda_proj}\cite{duda_lda}). 

\pastepic{Проекция двумерных точек разных классов на прямую}{design/lda_proj.png}{lda_proj}

Этот алгоритм обычно используется только для снижения размерности, но также его можно применить как классификатор (см. ниже).

Опишем алгоритм определения параметров этой прямой\cite{duda_lda}. Пусть $\mu_i$ - вектор средних величин признаков для класса $i$. Среднее значение спроецированных точек будет равно $\overline{m_i}=1/N_i\sum\limits_j w^tx_j$, где $N_i$ - кол-во точек в классе, $x_j$ - элементы класса. Также введём разброс проекции: $\overline{s_i^2}=\sum (w^tx_j-\overline{m_i})^2$. Тогда \defn{линейный дискриминант Фишера} для двух классов определяется как максимум функции $J(w)=\dfrac{|\overline{m_1}-\overline{m_2}|^2}{\overline{s_1^2}+\overline{s_2^2}}$.

Определим вспомогательные матрицы: $S_i=\sum(x-\mu_i)(x-\mu_i)^t$, $S_w=S_1+S_2$ - матрицы разброса. Так как $\overline{s_i^2}=\sum (w^tx_j-\overline{m_i})^2$, то $\overline{s_i^2}=w^tS_iw$, $\overline{s_1^2}+\overline{s_2^2}=w^tS_ww$. Также введём $S_B=(\mu_1-\mu_2)*(\mu_1-\mu_2)^t$ - матрицу разброса между классами; $(\overline{m_1}-\overline{m_2})^2=w^tS_Bw$. Тогда $J(w)=\dfrac{w^tS_Bw}{w^tS_ww}$. 

Для нахождения максимума этой функции продифференцируем полученное выражение по $w$ и приравняем производную к нулю. После упрощения выражения получим: $S_w^{-1}S_Bw=Jw$. Таким образом, $w$ должен быть собственным вектором для матрицы $S_w^{-1}S_B$. Так как $S_B=(\mu_1-\mu_2)*(\mu_1-\mu_2)^t$, то $w=S_w^{-1}(\mu_1-\mu_2)$. 

Полученная прямая $w$ является нормалью к гиперплоскости, которая оптимально разделяет проекции элементов двух классов. Процедура классификации заключается в сравнении значения проекции и некоторого порога; обычно его принимают равным $1/2(\mu_1+\mu_2)$.

Для выполнении мультиклассового распознавания применяется несколько техник. Во-первых, Rao\cite{lda_multiclass} обобщил алгоритм LDA для проекции не на прямую, а на подпространство размерности $C-1$, где $C$ - число классов. Во-вторых, можно использовать техники "один-против-всех" и "один-против-одного" для объединения результатов бинарной классификации.



% ссылки
% http://stu.alnam.ru/book_recob-45
% http://www.facweb.iitkgp.ernet.in/~sudeshna/courses/ML06/lda.pdf




\section{Алгоритмы, основанные на метриках расстояния}

Существует класс алгоритмов, основанных на вычислении расстояния (\defn{метрики}) между классифицируемыми объектами. 

\subsection{Алгоритм динамической трансформации временной шкалы (DTW)}

Основная идея алгоритма DTW - минимизация эффекта сдвигов и искажений сигналов во времени путём трансформации одного временного ряда в другой. Это позволяет находить сигналы схожих форм с разными фазами\cite{dtw_review}. 

Пусть даны два временных ряда $X=x_1,x_2...x_N$ и $Y=y_1,y_2...y_M$. Элементы рядов $x_i,y_j$ принадлежат одному пространству \missing{уточнить} $\Phi$, для которого введена метрика $d: \Phi \times \Phi \to R$. 

Алгоритм DTW заключается в следующем:

\begin{enumerate}
\item Строится матрица $D$, состоящая из значений метрик для всех возможных пар элементов рядов: $d_{ij}=|| x_i-y_j ||$. Эта метрика показывает величину трансформации одной точки в другую
\item Строится матрица трансформаций $T$:

$t_{ij}=d_{ij}+min(T_{i-1,j-1},T_{i-1,j},T_{i,j-1})$. Эта матрица позволяет найти \missing{уточнить} локальные оптимальные пути трансформаций для последовательностей соседних точек
\item Наконец, находится такая последовательность индексов $w_k=(i,j)_k$, что:
  \begin{itemize}
  \item начальный индекс - (1,1), конечный - (N,M)
  \item $i_{k+1}-i_k \leq 1, j_{k+1}-j_k \leq 1$ - условие непрерывности пути
  \item последовательности индексов $i, j$ не убывают (путь не возвращается в пройденные точки)
  \item суммарный вес $d_{ij}$ минимален среди всех допустимых вариантов
  \end{itemize}
\end{enumerate}

Результат работы алгоритма - оптимальный путь трансформации одного временного ряда в другой. Длина пути (суммарный вес элементов $d_k$, входящих в него), используется как метрика расстояния между рядами. 

\missing{для трёхмерных рядов - поправить код}

\missing{изображения}

\missing{дополнить}
Время работы алгоритма - $O(M*N)$
есть оптимизации


\subsection{Метод k ближайших соседей}

После вычисления метрик можно производить классификацию объектов, основываясь на расстоянии между ними. Одним из наиболее простых методов является алгоритм k ближайших соседей: объекту присваивается тот же класс, что имеет большинство его соседей (т.е. объектов, расстояние до которых минимально).

\missing{дописать}



\section{Контроль качества распознавания}

\subsection{Метрики качества}

Для оценивания качества распознавания используется \defn{матрица ошибок (confusion matrix)}. Это квадратная матрица $C_{N*N}$, где $N$ - количество классов. При классификации элемента тестовой выборки инкрементируется $c_{ij}$, где $i$ - номер ожидаемого класса, $j$ - номер результирующего класса. Таким образом, при идеальной классификации $C$ должна иметь вид диагональной матрицы, элементы которой - размеры классов тестовой выборки. При наличии ошибок элементы вне диагонали становятся ненулевыми.

%\missing{привести пример confmat}
% $Phi_i=\mxthree{\phi_{xx}}{\phi_{yx}}{\phi_{zx}}{\phi_{xy}}{\phi_{yy}}{\phi_{zy}}{\phi_{xz}}{\phi_{yz}}{\phi_{zz}}$, 

Наиболее простой метрикой качества является \defn{accuracy} - отношение числа правильных ответов классификатора к общему числу элементов:

$acc=\dfrac{trace(C)}{sum(C)}$, где $trace(C)$ - след матрицы (сумма диагональных элементов), $sum(C)$ - сумма всех элементов.

Эта метрика неприменима для классов разных размеров. К примеру, если в первом классе 10 элементов, а во втором - 100, то, назначая любому элементу второй класс, получим $acc=\dfrac{100}{10+100}=0,909$. Это достаточно высокое значение показателя качества, но такое правило классификации не имеет практического смысла.

Решение этой проблемы - учёт каждого класса по отдельности. Рассмотрим задачу бинарной классификации - необходимо выяснить, принадлежит ли объект классу или нет. Для неё матрица ошибок имеет следующий вид:

$\mxtwo{True~Positive~(TP) }{False~Negative~(FN)}{False~Positive~(FP)}{True~Negative~(TN)}$

Введём следующие показатели:

\begin{itemize}
\item \defn{точность (precision)}: $P=\dfrac{TP}{TP+FP}$ - показывает, сколько объектов среди распознанных как принадлежащие к классу реально принадлежат eму
\item \defn{полнота (recall)}: $R=\dfrac{TP}{TP+FN}$ - показывает долю объектов класса, распознанных правильно
\end{itemize}

В реальных системах достичь максимальных значений обоих параметров невозможно: если распознавать большинство объектов как принадлежащие классу, то из-за роста False Positive будет снижаться точность; в обратном же случае из-за роста False Negative будет снижаться полнота. Поэтому необходимо соблюдать баланс между ними с помощью метрики, которая учитывает оба этих показателя. Одна из таких метрик - т.н. \defn{F-мера (F-score)}:

$F_\beta=(1+\beta^2)\dfrac{P*R}{\beta^2*P+R}$, где $\beta$ - вес точности в метрике. Обычно принимают $\beta=1$, тогда формула приобретает вид среднего гармонического точности и полноты. 

Для мультиклассового распознавания необходимо учитывать метрику, вычисленную для каждого класса в отдельности. Для F-меры есть следующие варианты\cite{sklearn_metric}:

\begin{itemize}
\item macro-усреднение: точность и полнота по всем классам вычисляется как среднее арифметическое этих показателей по отдельным классам
\item micro-усреднение: точность и полнота по всем классам вычисляются по стандартным формулам, где $TP, FP, FN$ - суммы соответствующих показателей по отдельным классам
\end{itemize}

\subsection{Выбор параметров модели и гиперпараметров}

На результат работы любого алгоритма машинного обучения влияют следующие факторы:

\begin{itemize}
\item параметры модели - веса в нейронной сети, средние и дисперсии в гауссовом байесовском классификаторе. Определяются в процессе обучения модели
\item \defn{гиперпараметры} - не зависят от процесса обучения: количество и состав скрытых слоёв нейронной сети, функции активации и т.д. Эти параметры задаются перед началом обучения
\end{itemize}

Выбор значений гиперпараметров в общем случае сводится к перебору их возможных значений. Наиболее простой способ сделать это - разделить всё множество известных объектов на подмножества:

\begin{itemize}
\item обучающая выборка - используется для настройки параметров модели
\item контрольная выборка - используется для выбора гиперпараметров. Так как метрика на контрольной выборке не зависит от результатов обучения, это позволяет избежать переобучения на обучающей выборке
\item тестовая выборка - используется для получения итогового значения метрик качества. Она необходима для того, чтобы выбор оптимальных гиперпараметров не приводил к переобучению на контрольной выборке
\end{itemize}

Такое разбиение приводит к тому, что в подмножествах оказывается слишком мало элементов. Чтобы избежать этого, используют \defn{кросс-валидацию}. Её алгоритм заключается в следующем:

\begin{enumerate}
\item всё множество объектов делится на обучающую и тестовую выборку
\item из обучающей выборки тем или иным образом выбирается $k$ подмножеств (\defn{fold}; могут пересекаться между собой)
\item процедура обучения и валидации выполняется $k$ раз, при этом в качестве собственно обучающего множества используются $k-1$ из полученных подмножеств, а контроль осуществляется по оставшемуся
\item контроль обучения осуществлятся усреднением $k$ результатов
\end{enumerate}

Такой алгоритм избавляет от необходимости использовать отдельную контрольную выборку. Недостаток - повышение вычислительных затрат из-за $k$ повторений процедуры обучения.

Существует несколько подходов к выбору подмножеств:

\begin{itemize}
\item разбиение на $k$ непересекающихся подмножеств (\defn{k-fold})
\item leave-one-out - подвид k-fold, где $k$ равно размеру обучающей выборки
\item leave-p-out - перебираются все варианты с $p$ элементами в контрольной выборке
\item случайное разбиение
\end{itemize}

Для повышения качества обучения подмножества могут быть стратифицированы - отношение размеров классов в них будут равны таковому во всей обучающей выборке.

% https://habrahabr.ru/company/ods/blog/328372/
% http://scikit-learn.org/stable/modules/model_evaluation.html#multiclass-and-multilabel-classification