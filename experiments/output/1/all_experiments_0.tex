\begin{table}[\tableopts]
\begin{tabular}{\tableformat}
 \hline{} & Подтягивания & Отжимания & Приседания & Ходьба \\ \hline
Подтягивания & 39 & 0 & 0 & 1 \\ \hline
Отжимания & 5 & 32 & 0 & 1 \\ \hline
Приседания & 0 & 0 & 47 & 4 \\ \hline
Ходьба & 0 & 1 & 0 & 71 \\ \hline
\multicolumn{5}{|c|}{Точность: 0.940299} \\ \hline
\multicolumn{5}{|c|}{Значение F-метрики: 0.935546} \\ \hline
\multicolumn{5}{|c|}{Время обучения: 7.000658 с} \\ \hline
\multicolumn{5}{|c|}{Время классификации: 0.377694 с} \\ \hline
\end{tabular}
\caption{\label{table:full_WaveletsFeaturesExtractor_MLPClassifier} Выделение коэффициентов дискретного-вейвлет преобразования, применение нейронной сети прямого распространения}
\end{table}

\begin{table}[\tableopts]
\begin{tabular}{\tableformat}
 \hline{} & Подтягивания & Отжимания & Приседания & Ходьба \\ \hline
Подтягивания & 39 & 0 & 0 & 1 \\ \hline
Отжимания & 6 & 32 & 0 & 0 \\ \hline
Приседания & 0 & 0 & 47 & 4 \\ \hline
Ходьба & 0 & 2 & 1 & 69 \\ \hline
\multicolumn{5}{|c|}{Точность: 0.930348} \\ \hline
\multicolumn{5}{|c|}{Значение F-метрики: 0.925309} \\ \hline
\multicolumn{5}{|c|}{Время обучения: 6.616508 с} \\ \hline
\multicolumn{5}{|c|}{Время классификации: 0.152852 с} \\ \hline
\end{tabular}
\caption{\label{table:full_RawExtractor_MLPClassifier} Использование значений ряда как признаков, применение нейронной сети прямого распространения}
\end{table}

\begin{table}[\tableopts]
\begin{tabular}{\tableformat}
 \hline{} & Подтягивания & Отжимания & Приседания & Ходьба \\ \hline
Подтягивания & 39 & 0 & 0 & 1 \\ \hline
Отжимания & 5 & 32 & 0 & 1 \\ \hline
Приседания & 0 & 0 & 47 & 4 \\ \hline
Ходьба & 1 & 4 & 1 & 66 \\ \hline
\multicolumn{5}{|c|}{Точность: 0.915423} \\ \hline
\multicolumn{5}{|c|}{Значение F-метрики: 0.912168} \\ \hline
\multicolumn{5}{|c|}{Время обучения: 6.861761 с} \\ \hline
\multicolumn{5}{|c|}{Время классификации: 0.276236 с} \\ \hline
\end{tabular}
\caption{\label{table:full_SignalInterpolator_MLPClassifier} Использование коэффициентов аппроксимирующих сплайнов как признаков, применение нейронной сети прямого распространения}
\end{table}

\begin{table}[\tableopts]
\begin{tabular}{\tableformat}
 \hline{} & Подтягивания & Отжимания & Приседания & Ходьба \\ \hline
Подтягивания & 39 & 0 & 0 & 1 \\ \hline
Отжимания & 5 & 31 & 0 & 2 \\ \hline
Приседания & 0 & 0 & 45 & 6 \\ \hline
Ходьба & 0 & 3 & 1 & 68 \\ \hline
\multicolumn{5}{|c|}{Точность: 0.910448} \\ \hline
\multicolumn{5}{|c|}{Значение F-метрики: 0.907567} \\ \hline
\multicolumn{5}{|c|}{Время обучения: 28.061538 с} \\ \hline
\multicolumn{5}{|c|}{Время классификации: 6.477979 с} \\ \hline
\end{tabular}
\caption{\label{table:full_HMMABOutExtractor_MLPClassifier} Выделение параметров скрытой марковской модели, применение нейронной сети прямого распространения}
\end{table}

\begin{table}[\tableopts]
\begin{tabular}{\tableformat}
 \hline{} & Подтягивания & Отжимания & Приседания & Ходьба \\ \hline
Подтягивания & 39 & 0 & 0 & 1 \\ \hline
Отжимания & 5 & 32 & 0 & 1 \\ \hline
Приседания & 0 & 0 & 41 & 10 \\ \hline
Ходьба & 1 & 6 & 3 & 62 \\ \hline
\multicolumn{5}{|c|}{Точность: 0.865672} \\ \hline
\multicolumn{5}{|c|}{Значение F-метрики: 0.868056} \\ \hline
\multicolumn{5}{|c|}{Время обучения: 0.690104 с} \\ \hline
\multicolumn{5}{|c|}{Время классификации: 0.142761 с} \\ \hline
\end{tabular}
\caption{\label{table:full_WaveletsFeaturesExtractor_GaussianNB} Выделение коэффициентов дискретного-вейвлет преобразования, применение наивного байесовского классификатора}
\end{table}

\begin{table}[\tableopts]
\begin{tabular}{\tableformat}
 \hline{} & Подтягивания & Отжимания & Приседания & Ходьба \\ \hline
Подтягивания & 39 & 0 & 0 & 1 \\ \hline
Отжимания & 5 & 25 & 0 & 8 \\ \hline
Приседания & 0 & 0 & 42 & 9 \\ \hline
Ходьба & 0 & 9 & 8 & 55 \\ \hline
\multicolumn{5}{|c|}{Точность: 0.800995} \\ \hline
\multicolumn{5}{|c|}{Значение F-метрики: 0.803330} \\ \hline
\multicolumn{5}{|c|}{Время обучения: 29.315519 с} \\ \hline
\multicolumn{5}{|c|}{Время классификации: 7.321875 с} \\ \hline
\end{tabular}
\caption{\label{table:full_HMMABOutExtractor_GaussianNB} Выделение параметров скрытой марковской модели, применение наивного байесовского классификатора}
\end{table}

\begin{table}[\tableopts]
\begin{tabular}{\tableformat}
 \hline{} & Подтягивания & Отжимания & Приседания & Ходьба \\ \hline
Подтягивания & 30 & 9 & 0 & 1 \\ \hline
Отжимания & 6 & 28 & 1 & 3 \\ \hline
Приседания & 2 & 1 & 41 & 7 \\ \hline
Ходьба & 2 & 2 & 4 & 64 \\ \hline
\multicolumn{5}{|c|}{Точность: 0.810945} \\ \hline
\multicolumn{5}{|c|}{Значение F-метрики: 0.796014} \\ \hline
\multicolumn{5}{|c|}{Время обучения: 6.637767 с} \\ \hline
\multicolumn{5}{|c|}{Время классификации: 0.308705 с} \\ \hline
\end{tabular}
\caption{\label{table:full_SpectrumInterpolator_MLPClassifier} Использование коэффициентов сплайнов, аппроксимирующих спектр, как признаков, применение нейронной сети прямого распространения}
\end{table}

\begin{table}[\tableopts]
\begin{tabular}{\tableformat}
 \hline{} & Подтягивания & Отжимания & Приседания & Ходьба \\ \hline
Подтягивания & 39 & 0 & 0 & 1 \\ \hline
Отжимания & 6 & 22 & 1 & 9 \\ \hline
Приседания & 0 & 0 & 43 & 8 \\ \hline
Ходьба & 0 & 10 & 6 & 56 \\ \hline
\multicolumn{5}{|c|}{Точность: 0.796020} \\ \hline
\multicolumn{5}{|c|}{Значение F-метрики: 0.791207} \\ \hline
\multicolumn{5}{|c|}{Время обучения: 25.172445 с} \\ \hline
\multicolumn{5}{|c|}{Время классификации: 6.282848 с} \\ \hline
\end{tabular}
\caption{\label{table:full_HMMABOutExtractor_LinearDiscriminantAnalysis} Выделение параметров скрытой марковской модели, применение линейного дискриминантного анализа}
\end{table}

\begin{table}[\tableopts]
\begin{tabular}{\tableformat}
 \hline{} & Подтягивания & Отжимания & Приседания & Ходьба \\ \hline
Подтягивания & 39 & 0 & 0 & 1 \\ \hline
Отжимания & 5 & 30 & 0 & 3 \\ \hline
Приседания & 0 & 0 & 41 & 10 \\ \hline
Ходьба & 0 & 17 & 8 & 47 \\ \hline
\multicolumn{5}{|c|}{Точность: 0.781095} \\ \hline
\multicolumn{5}{|c|}{Значение F-метрики: 0.790305} \\ \hline
\multicolumn{5}{|c|}{Время обучения: 0.347014 с} \\ \hline
\multicolumn{5}{|c|}{Время классификации: 0.092299 с} \\ \hline
\end{tabular}
\caption{\label{table:full_SignalInterpolator_GaussianNB} Использование коэффициентов аппроксимирующих сплайнов как признаков, применение наивного байесовского классификатора}
\end{table}

\begin{table}[\tableopts]
\begin{tabular}{\tableformat}
 \hline{} & Подтягивания & Отжимания & Приседания & Ходьба \\ \hline
Подтягивания & 30 & 8 & 0 & 2 \\ \hline
Отжимания & 7 & 28 & 0 & 3 \\ \hline
Приседания & 1 & 3 & 40 & 7 \\ \hline
Ходьба & 2 & 3 & 4 & 63 \\ \hline
\multicolumn{5}{|c|}{Точность: 0.800995} \\ \hline
\multicolumn{5}{|c|}{Значение F-метрики: 0.787312} \\ \hline
\multicolumn{5}{|c|}{Время обучения: 2.341922 с} \\ \hline
\multicolumn{5}{|c|}{Время классификации: 0.109339 с} \\ \hline
\end{tabular}
\caption{\label{table:full_FFTCoeffsExtractor_MLPClassifier} Выделение коэффициентов быстрого преобразования Фурье, применение нейронной сети прямого распространения}
\end{table}