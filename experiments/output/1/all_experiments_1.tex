\begin{table}[\tableopts]
\begin{tabular}{\tableformat}
 \hline{} & Подтягивания & Отжимания & Приседания & Ходьба \\ \hline
Подтягивания & 39 & 0 & 0 & 1 \\ \hline
Отжимания & 6 & 26 & 0 & 6 \\ \hline
Приседания & 0 & 2 & 41 & 8 \\ \hline
Ходьба & 1 & 15 & 6 & 50 \\ \hline
\multicolumn{5}{|c|}{Точность: 0.776119} \\ \hline
\multicolumn{5}{|c|}{Значение F-метрики: 0.778903} \\ \hline
\multicolumn{5}{|c|}{Время обучения: 0.397135 с} \\ \hline
\multicolumn{5}{|c|}{Время классификации: 0.112066 с} \\ \hline
\end{tabular}
\caption{\label{table:full_SignalInterpolator_LinearDiscriminantAnalysis} Использование коэффициентов аппроксимирующих сплайнов как признаков, применение линейного дискриминантного анализа}
\end{table}

\begin{table}[\tableopts]
\begin{tabular}{\tableformat}
 \hline{} & Подтягивания & Отжимания & Приседания & Ходьба \\ \hline
Подтягивания & 39 & 0 & 0 & 1 \\ \hline
Отжимания & 6 & 26 & 0 & 6 \\ \hline
Приседания & 0 & 2 & 41 & 8 \\ \hline
Ходьба & 1 & 15 & 6 & 50 \\ \hline
\multicolumn{5}{|c|}{Точность: 0.776119} \\ \hline
\multicolumn{5}{|c|}{Значение F-метрики: 0.778903} \\ \hline
\multicolumn{5}{|c|}{Время обучения: 0.205030 с} \\ \hline
\multicolumn{5}{|c|}{Время классификации: 0.071050 с} \\ \hline
\end{tabular}
\caption{\label{table:full_RawExtractor_LinearDiscriminantAnalysis} Использование значений ряда как признаков, применение линейного дискриминантного анализа}
\end{table}

\begin{table}[\tableopts]
\begin{tabular}{\tableformat}
 \hline{} & Подтягивания & Отжимания & Приседания & Ходьба \\ \hline
Подтягивания & 39 & 0 & 0 & 1 \\ \hline
Отжимания & 5 & 27 & 0 & 6 \\ \hline
Приседания & 0 & 0 & 40 & 11 \\ \hline
Ходьба & 0 & 17 & 8 & 47 \\ \hline
\multicolumn{5}{|c|}{Точность: 0.761194} \\ \hline
\multicolumn{5}{|c|}{Значение F-метрики: 0.770330} \\ \hline
\multicolumn{5}{|c|}{Время обучения: 0.186458 с} \\ \hline
\multicolumn{5}{|c|}{Время классификации: 0.055921 с} \\ \hline
\end{tabular}
\caption{\label{table:full_RawExtractor_GaussianNB} Использование значений ряда как признаков, применение наивного байесовского классификатора}
\end{table}

\begin{table}[\tableopts]
\begin{tabular}{\tableformat}
 \hline{} & Подтягивания & Отжимания & Приседания & Ходьба \\ \hline
Подтягивания & 35 & 4 & 0 & 1 \\ \hline
Отжимания & 17 & 20 & 0 & 1 \\ \hline
Приседания & 0 & 0 & 41 & 10 \\ \hline
Ходьба & 3 & 4 & 6 & 59 \\ \hline
\multicolumn{5}{|c|}{Точность: 0.771144} \\ \hline
\multicolumn{5}{|c|}{Значение F-метрики: 0.751203} \\ \hline
\multicolumn{5}{|c|}{Время обучения: 1.057427 с} \\ \hline
\multicolumn{5}{|c|}{Время классификации: 0.335710 с} \\ \hline
\end{tabular}
\caption{\label{table:full_STFTCoeffsExtractor_LinearDiscriminantAnalysis} Выделение коэффициентов оконного преобразования Фурье, применение линейного дискриминантного анализа}
\end{table}

\begin{table}[\tableopts]
\begin{tabular}{\tableformat}
 \hline{} & Подтягивания & Отжимания & Приседания & Ходьба \\ \hline
Подтягивания & 22 & 16 & 0 & 2 \\ \hline
Отжимания & 16 & 21 & 1 & 0 \\ \hline
Приседания & 0 & 0 & 46 & 5 \\ \hline
Ходьба & 2 & 1 & 1 & 68 \\ \hline
\multicolumn{5}{|c|}{Точность: 0.781095} \\ \hline
\multicolumn{5}{|c|}{Значение F-метрики: 0.739274} \\ \hline
\multicolumn{5}{|c|}{Время обучения: 7.493804 с} \\ \hline
\multicolumn{5}{|c|}{Время классификации: 0.574086 с} \\ \hline
\end{tabular}
\caption{\label{table:full_STFTCoeffsExtractor_MLPClassifier} Выделение коэффициентов оконного преобразования Фурье, применение нейронной сети прямого распространения}
\end{table}

\begin{table}[\tableopts]
\begin{tabular}{\tableformat}
 \hline{} & Подтягивания & Отжимания & Приседания & Ходьба \\ \hline
Подтягивания & 26 & 14 & 0 & 0 \\ \hline
Отжимания & 11 & 26 & 0 & 1 \\ \hline
Приседания & 3 & 1 & 40 & 7 \\ \hline
Ходьба & 6 & 6 & 0 & 60 \\ \hline
\multicolumn{5}{|c|}{Точность: 0.756219} \\ \hline
\multicolumn{5}{|c|}{Значение F-метрики: 0.738170} \\ \hline
\multicolumn{5}{|c|}{Время обучения: 0.540405 с} \\ \hline
\multicolumn{5}{|c|}{Время классификации: 0.129386 с} \\ \hline
\end{tabular}
\caption{\label{table:full_SpectrumInterpolator_LinearDiscriminantAnalysis} Использование коэффициентов сплайнов, аппроксимирующих спектр, как признаков, применение линейного дискриминантного анализа}
\end{table}

\begin{table}[\tableopts]
\begin{tabular}{\tableformat}
 \hline{} & Подтягивания & Отжимания & Приседания & Ходьба \\ \hline
Подтягивания & 26 & 14 & 0 & 0 \\ \hline
Отжимания & 11 & 26 & 0 & 1 \\ \hline
Приседания & 3 & 1 & 40 & 7 \\ \hline
Ходьба & 6 & 6 & 0 & 60 \\ \hline
\multicolumn{5}{|c|}{Точность: 0.756219} \\ \hline
\multicolumn{5}{|c|}{Значение F-метрики: 0.738170} \\ \hline
\multicolumn{5}{|c|}{Время обучения: 0.440216 с} \\ \hline
\multicolumn{5}{|c|}{Время классификации: 0.098641 с} \\ \hline
\end{tabular}
\caption{\label{table:full_FFTCoeffsExtractor_LinearDiscriminantAnalysis} Выделение коэффициентов быстрого преобразования Фурье, применение линейного дискриминантного анализа}
\end{table}

\begin{table}[\tableopts]
\begin{tabular}{\tableformat}
 \hline{} & Подтягивания & Отжимания & Приседания & Ходьба \\ \hline
Подтягивания & 39 & 0 & 0 & 1 \\ \hline
Отжимания & 6 & 22 & 0 & 10 \\ \hline
Приседания & 0 & 1 & 38 & 12 \\ \hline
Ходьба & 1 & 17 & 6 & 48 \\ \hline
\multicolumn{5}{|c|}{Точность: 0.731343} \\ \hline
\multicolumn{5}{|c|}{Значение F-метрики: 0.735602} \\ \hline
\multicolumn{5}{|c|}{Время обучения: 0.781841 с} \\ \hline
\multicolumn{5}{|c|}{Время классификации: 0.236713 с} \\ \hline
\end{tabular}
\caption{\label{table:full_WaveletsFeaturesExtractor_LinearDiscriminantAnalysis} Выделение коэффициентов дискретного-вейвлет преобразования, применение линейного дискриминантного анализа}
\end{table}

\begin{table}[\tableopts]
\begin{tabular}{\tableformat}
 \hline{} & Подтягивания & Отжимания & Приседания & Ходьба \\ \hline
Подтягивания & 31 & 9 & 0 & 0 \\ \hline
Отжимания & 22 & 14 & 0 & 2 \\ \hline
Приседания & 0 & 1 & 45 & 5 \\ \hline
Ходьба & 8 & 9 & 1 & 54 \\ \hline
\multicolumn{5}{|c|}{Точность: 0.716418} \\ \hline
\multicolumn{5}{|c|}{Значение F-метрики: 0.687023} \\ \hline
\multicolumn{5}{|c|}{Время обучения: 0.570833 с} \\ \hline
\multicolumn{5}{|c|}{Время классификации: 0.128566 с} \\ \hline
\end{tabular}
\caption{\label{table:full_SpectrumInterpolator_GaussianNB} Использование коэффициентов сплайнов, аппроксимирующих спектр, как признаков, применение наивного байесовского классификатора}
\end{table}

\begin{table}[\tableopts]
\begin{tabular}{\tableformat}
 \hline{} & Подтягивания & Отжимания & Приседания & Ходьба \\ \hline
Подтягивания & 25 & 13 & 1 & 1 \\ \hline
Отжимания & 10 & 24 & 1 & 3 \\ \hline
Приседания & 4 & 5 & 38 & 4 \\ \hline
Ходьба & 6 & 6 & 7 & 53 \\ \hline
\multicolumn{5}{|c|}{Точность: 0.696517} \\ \hline
\multicolumn{5}{|c|}{Значение F-метрики: 0.679719} \\ \hline
\multicolumn{5}{|c|}{Время обучения: 195.091784 с} \\ \hline
\multicolumn{5}{|c|}{Время классификации: 48.137033 с} \\ \hline
\end{tabular}
\caption{\label{table:full_DTWTransformer_KNeighborsClassifier} Применение алгоритма динамического преобразования времени для определения расстояния между рядами, применение метода k ближайших соседей}
\end{table}