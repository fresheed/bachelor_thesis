\chapter{Заключение}

В ходе работы исследованы различные методы выделения признаков и классификации, применимые к задаче распознавания двигательной активности. Движения представлялись трёхмерными временными рядами - данными с акселерометра носимого на запястье устройства.

Изученные алгоритмы различаются, в первую очередь, точностью классификации, которая достигается с их применением. Однако, учитывая то, что распознавание производится в реальном времени, следует учитывать также время исполнения алгоритма; некоторые алгоритмы неприменимы в реальных условиях, потому что не удовлетворяют этому условию.

Условно исследованные алгоритмы классификации можно поделить на работающие с признаковым описанием и с метрикой в пространстве объектов. Среди последних был исследован только алгоритм динамического преобразования временной шкалы (DTW). Несмотря на то, что он давал достаточно высокую точность, из-за большого времени преобразования его нельзя применятоь в режиме реального времени. Тем не менее, он применим для обработки данных после окончания тренировки. Метрика, получаемая с помощью DTW, в дальнейшем использовалась в алгоритме классификации методом ближайших соседей.

Для выделения признаков может использоваться временное или частотное представление сигнала. Для получения признаков из временной области использовались:
\begin{itemize}
\item методы авторегрессии. В качестве признаков использовались коэффициенты уравнений, описывающих сигнал. Данные методы можно считать наименее подходящими к данной задаче: точность была одной из самых низких, а время распознавания - достаточно большим
\item скрытые марковские модели. В качестве признаков использовались коэффициенты матриц, описывающих модель, а также параметры распределений, описывающих выходные (наблюдаемые) состояния. Точность была достаточно высокой, время обучения - достаточно большим
\item аппроксимация сигнала сплайнами. Признаки - коэффициенты сплайнов. Достаточно быстрый алгоритм, дающий очень высокую точность. Недостаток - классифицируемые должны иметь одинаковую длину
\end{itemize}

В частотной области использовались методы, основанные на преобразовании Фурье:
\begin{itemize}
\item непосредственное использование коэффициентов Фурье
\item оконное преобразование Фурье (признаки - последовательность групп коэффициентов, полученных для каждого из положений окна)
\item вейвлет-преобразование (аналогично предыдущему)
\item аппроксимация спектра сплайнами
\end{itemize}
Все частотные методы показывали более высокую точность и малое время работы, чем временные методы, поэтому их можно считать более перспективными. Общий недостаток - необходимость соблюдения одной и той же длины сигналов для того, чтобы коэффициенты Фурье соответствовали одним и тем же частотам.

Среди алгоритмов классификации использовались нейронные сети, наивный байесовский классификатор и линейный дискриминантный анализ. Наиболее высокую точность давали нейронные сети, наименьшее время работы - байесовский классификатор.

Для определения гиперпараметров алгоритмов (число нейронов, порядок авторегрессионной модели и т.д.) использовалась кросс-валидация по 5 разбиениям. В ходе экспериментов было установлено, что из-за несовпадающих размеров классов использование accuracy давало не оптимальный результат - система чаще склонялась к выбору большего класса, поэтому в качестве оптимизируемого критерия использовалась F-мера.

\missing{вывод по реализации} 
%py vs java

В результате работы разработаны приложения для умных часов и смартфона на Android, позволяющие в реальном времени распознавать текущую активность.

\missing{добавить конкретные цифры к методам}

\missing{на будущее - см tex}
% перенос на java
% использование только смартфона
% ещё большая автоматизация сборки и деплоя
% дообучение

\missing{уточнить выводы}