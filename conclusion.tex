\conclusion

В ходе работы были исследованы различные методы выделения признаков и классификации, применимые к задаче распознавания двигательной активности. Движения представлялись многомерными временными рядами - данными с акселерометра носимого на запястье устройства.

В результате работы разработаны приложения для умных часов и смартфона на Android, позволяющие в реальном времени распознавать текущую активность. Выполнены основные требования - обработка данных производится в реальном времени, классификация успешно выполняется для разных пользователей. 

Разработанное приложение предлагает решение проблемы распознавания активности в реальном времени, которая на данный момент недостаточно разработана. По сравнению с существующими решениями приложение имеет преимущество - возможность выбора алгоритма классификации, который лучше всего подходит к тренировке и к конкретному пользователю. Исходный код для проектов на Java и Python расположен в открытых репозиториях \cite{repo_java} и \cite{repo_python} соответственно.

Для определения гиперпараметров алгоритмов (число нейронов, порядок авторегрессионной модели и т.д.) использовалась кросс-валидация по 5 разбиениям. В ходе экспериментов было установлено, что из-за несовпадающих размеров классов использование точности в качестве метрики качества давало не оптимальный результат - система чаще склонялась к выбору большего класса, поэтому в качестве оптимизируемого критерия использовалась F-мера. Тем не менее, точность остаётся показателем, наглядно отображающем результат работы алгоритма, поэтому в работе часто приводятся именно её значения.

В ходе работы было поставлено несколько экспериментов с целью выбрать оптимальные алгоритмы, которые работали бы в различных условиях. Главный вывод из результатов экспериментов - для ситуаций, когда пользователь участвовал в создании обучающей выборки и не участвовал, лучше использовать разные алгоритмы - тогда качество распознавания будет выше.

Алгоритмами выделения признаков, наиболее подходящими к задаче, оказались:
\begin{itemize}
\item дискретное вейвлет-преобразование
\item представление временных рядов непосредственно как набора признаков
\item аппроксимация сигнала сплайнами
\item построение скрытых марковских моделей
\end{itemize}


Приоритетные направления дальнейшей работы: 

\begin{itemize}
\item возможность дообучения модели для конкретного пользователя и добавление новых упражнений
\item полная автоматизация сборки клиентского приложения - предварительный запуск экспериментов и перенос обученных моделей непосредственно в процедуру сборки Android-приложения
\item реализация всей системы на единой платформе (Python/Java), что упростит взаимодействие между её частями
\item получение данных исключительно с сенсора смартфона - это сделало бы систему распознавания ещё доступнее
\end{itemize}

