\conclusion

В ходе работы исследованы различные методы выделения признаков и классификации, применимые к задаче распознавания двигательной активности. Движения представлялись многомерными временными рядами - данными с акселерометра носимого на запястье устройства.

В результате работы разработаны приложения для умных часов и смартфона на Android, позволяющие в реальном времени распознавать текущую активность. Выполнены основные требования - обработка данных производится в реальном времени, классификация успешно выполняется для разных пользователей. Разработанное решение вносит новизну в пока ещё слабо разработанную область приложений для контроля тренировки в реальном времени. Исходный код для проектов на Java и Python расположен в открытых репозиториях \cite{repo_java} и \cite{repo_python} соответственно.

Практическая значимость работы - возможнось применения приложения рядовыми пользователями без дополнительных настроек. По сравнению с существующими решениями приложение имеет следующие преимущества: нет необходимости связываться с центральным сервером, \missing{что ещё?}.

Для определения гиперпараметров алгоритмов (число нейронов, порядок авторегрессионной модели и т.д.) использовалась кросс-валидация по 5 разбиениям. В ходе экспериментов было установлено, что из-за несовпадающих размеров классов использование точности в качестве метрики качества давало не оптимальный результат - система чаще склонялась к выбору большего класса, поэтому в качестве оптимизируемого критерия использовалась F-мера.

Алгоритмами, наиболее подходящими к задаче, оказались дискретное вейвлет-преобразование и представление временных рядов непосредственно как набора признаков. Скрытые марковские модели обеспечили бОльшую точность, но потребовали намного больше времени на выделение признаков. Для классификации лучше всего подошёл аппарат нейронных сетей. Максимальная точность распознавания - 91.9\% - была достигнута с применением вейвлет-преобразования и нейронных сетей.

Приоритетные направления дальнейшей работы: 

\begin{itemize}
\item возможность дообучения модели для конкретного пользователя и добавление новых упражнений
\item полная автоматизация сборки клиентского приложения - предварительный запуск экспериментов и перенос обученных моделей непосредственно в процедуру сборки Android-приложения
\item реализация всей системы только на одном языке, что упростит взаимодействие между её частями
\item работа с сенсором смартфона - это сделало бы систему распознавания ещё доступнее
\end{itemize}

