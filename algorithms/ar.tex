\subsection{Классификация на основе коэффициентов авторегрессии}

Модели авторегрессии широко используются в статистике для анализа и прогнозирования стационарных рядов. В их основе лежит представление временного ряда как выхода линейного фильтра, на вход которого подана последовательность нормально распределённых случайных величин:

$z_t=\mu+a_t+\phi_1a_{t-1}+\phi_2a_{t-2}...=\mu+\Phi(B)a_t$, 

где $z_t$ - элемент ряда с индексом $t$, $\mu$ - среднее значение (уровень) ряда, $a_i$ - случайная величина, действующая в момент времени $i$, $\Psi(B)=1+\psi B+\psi_2 B^2+...$ - передаточная функция фильтра, $B$ - лаговый оператор: $Bz_t=z_{t-1}$.

На практике используются модели, представляющие ряд в виде регрессии от конечного числа величин\cite{bj_ts}: 

\begin{itemize}
\item модель авторегресии (\defn{AR-модель}): 

$z_t=\mu+\sum_{i=1}^p \phi_iB^i z_t + a_t$. В данном случае очередное значение ряда линейно зависит от предыдущих $p$ значений ряда
\item модель скользящего среднего (\defn{MA-модель}):

$z_t=\mu + a_t + \sum_{i=1}^q -\theta_iB^i a_t $. В данном случае очередное значение ряда линейно зависит от $q$ предыдущих случайных величин
\item модель авторегрессии-случайного среднего (\defn{ARMA-модель}) представляет собой комбинацию указанных выше методов:

$z_t=\mu + a_t + \sum_{i=1}^p \phi_iB^i z_t + \sum_{i=1}^q -\theta_iB^i a_t $. Применение этой модели позволяет снизить суммарное число параметров, что упрощает дальнейший анализ
\end{itemize}

В качестве признаков объекта могут использоваться параметры $\phi_i, \theta_i, \mu$, дисперсия случайной величины $a_i$ - $\sigma^2$, а также порядки моделей.

Обратим внимание на то, что в решаемой задаче временные ряды являются многомерным. Необходимо каким-либо образом учитывать все их измерения:

\begin{itemize}
\item Наиболее простой вариант - предположить, что временные ряды в каждом измерении независимы между собой. В этом случае их можно анализировать по отдельности, а общая модель будет описываться конкатенацией параметров отдельных измерений.
\item Учесть зависимости между рядами отдельных измерений. Для методов авторегрессии это, в частности, означает, что модель теперь представляется следующим образом (на примере AR-модели):

$U_t=M+\sum_{i=1}^p \Phi_iB^i U_t + A_t$,  где:

 $U_t=\colthree{x_t}{y_t}{z_t}$, $Phi_i=\mxthree{\phi_{xx}}{\phi_{yx}}{\phi_{zx}}{\phi_{xy}}{\phi_{yy}}{\phi_{zy}}{\phi_{xz}}{\phi_{yz}}{\phi_{zz}}$, 
 $A_t=\colthree{a_x}{a_y}{a_z}$

Модели, реализующие это, называются VAR(VMA,VARMA)-моделями. 
\end{itemize}

\missing{алгоритмы для определения к-тов АР}