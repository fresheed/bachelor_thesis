\subsection{Классификация на основе характеристик сигнала в частотной области}

Рассмотренные выше методы пользовались представлением сигнала как последовательности векторов. Между тем, большую информацию может предоставить спектр сигнала. Для его получения воспользуемся дискретным преобразованием Фурье (\defn{ДПФ}):

$S_k=\sum_{i=0}^{N-1}s_i*exp(-j\dfrac{2\pi}{N}ik), k=0..N-1$, где $S_k$ - комплексные коэффициенты Фурье, $s_i$ - отсчёты сигнала.

\missing{нужно ли приводить алгоритм ДПФ?}

Видно, что сложность расчёта спектра - $O(N^2)$ (суммирование $N$ точек сигнала для каждой из $N$ точек спектра). Для ускорения расчётов применяют подвид ДПФ - быстрое преобразование Фурье (\defn{БПФ}). Оно основано на том, что выполнение БПФ над $2M$ точками можно свести к двум независимым БПФ размерности $M$ и их объединению. 

Приведём один из таких алгоритмов - прореживание по времени \cite{fft_alg}. Пусть $y_i=s_{2i}$ и $z_i=s_{2i+1}$ - последовательности из чётных и нечётных элементов исходного ряда соответственно. Тогда

$S_k=\sum_{i=0}^{N/2-1}y_i*exp(-j\dfrac{2\pi}{N}2ik)+\sum_{i=0}^{N/2-1}z_i*exp(-j\dfrac{2\pi}{N}(2i+1)k)=\sum_{i=0}^{N/2-1}y_i*exp(-j\dfrac{2\pi}{N/2}2k)+exp(-j\dfrac{2\pi}{N}k)\sum_{i=0}^{N/2-1}z_i*exp(-j\dfrac{2\pi}{N/2}ik)$.

Таким образом, получили два ДПФ размерности $N/2$. Искомое ДПФ выражается как 
$S_k=Y_k+exp(-j\dfrac{2\pi}{N}k)Z_k$. Так как ДПФ размерности $N/2$ даёт $N/2$ коэффициентов, воспользуемся свойством периодичности спектра: $Y_{i+N/2}=Y_i, Z_{i+N/2}=Z_i$; следовательно, коэффициенты $S_{N/2}..S_{N}$ будут равны соответственно $S_{1}..S_{N/2-1}$. 

Схему, выполняющую соответствующие вычисления, можно представить следующим образом:

\pastepic{Вычисление N-точечного ДПФ через два N/2-точечных}{fft.png}

На каждое половинное ДПФ выполняется $N^2/4$ операций; кроме того, производится $N/2$ умножений на комплексную экспоненту. В итоге такой алгоритм требует $2*N^2/4+N/2=N(N+1)/2$ операций - практически вдвое меньше, чем обычное ДПФ. Если $N$ является степенью двойки, то такое деление можно выполнять на каждом шаге и, таким образом, свести число операций к $N*log_2(N)$, что при больших $N$ даёт существенный прирост производительности.

В качестве выделяемых признаков можно использовать непосредственно спектральные коэффициенты или другие характеристики частотной области, которые можно из них получить.

\missing{stft}