
\section{Требования к системе распознавания}

Назначение системы в целом - по данным с сенсора умных часов распознавать вид текущей двигательной активности и отображать его пользователю в реальном времени. Для этого разрабатывается клиентское приложение для ОС Android. Выбор ОС обусловлен тем, что это наиболее распространённая операционная система для мобильных устройств. 

Так как процесс обучения может длиться долго и требовать большого количества ресурсов, его целесообразно проводить в офлайн-режиме, а в клиентское приложение загружать уже обученную модель. Достоинство такой модели - упрощение клиентской части; недостаток - необходимость загрузки параметров модели в приложение.

Требования к клиентскому приложению:

\begin{itemize}
\item приложение должно получать и обрабатывать данные с умных часов в реальном времени
\item должны поддерживаться следующие виды активности: ходьба, отжимания на брусьях, подтягивания, приседания
\item управление приложением должно осуществляться исключительно с помощью смартфона
\item приложение должно эффективно работать с разными пользователями
\item необходима возможность выбирать алгоритм классификации для наилучшей адаптации к виду тренировки и конкретному пользователю
\end{itemize}

Требования к устройству:
\begin{itemize}
\item Клиентское приложение требует версию Android не ниже 4.0.3
\item Приложение для часов требует ОС Android Wear версии не ниже 7.1. Для сбора данных используется встроенный акселерометр
\end{itemize}
