
\section{Архитектура системы}

Как было показано на рис.\ref{fig:system}, система распознавания состоит из нескольких частей. Наиболее общее деление - на онлайн-часть (Android-приложения и сопутствующий код на Java) и офлайн-часть (код на Python, обучающий модели). Отправка данных в офлайн-систему осуществляется путём загрузки соответствующих файлов в хранилище Dropbox. Загрузка моделей в приложение на данном этапе осуществляется вручную - файл с параметрами модели включается в apk-архив с приложением, и в момент создания классификатора он инициализируется этими параметрами.

Одной из проблем такой архитектуры является использование обученных моделей в клиентском приложении. Так как в разных частях системы используются разные языки и среды выполнения, нельзя напрямую использовать объект с обученной моделью в приложении на Android. Были рассмотрены следующие решения:

\begin{itemize}
\item реализовать приложение на Android на каком-либо Python-фреймворке, например, Kivy. Такие фреймворки в теории позволяют полностью отказаться от Java-кода. Проблема в том, что sklearn имеет среди зависимостей компилируемые библиотеки, а их перенос на Android - достаточно трудоёмкая задача. Тот же Kivy предлагает интерфейс "рецептов" (recipes) для компилирования зависимостей, но для sklearn на момент написания работы его не существовало
\item отказаться от Python-кода в офлайн-части и сразу тренировать модели на Java. Тогда модель можно было бы легко перенести в клиентское приложение. От этого решения пришлось отказаться, так как scikit-learn предоставляет большое количество уже реализованных алгоритмов, удобный интерфейс для реализации новых и множество вспомогательных инструментов (например, оптимизация гиперпараметров). Кроме того, Python де-факто, наравне с R, является основным языком для data mining, поэтому для него доступно большое количество полезных библиотек (например, pandas). Всё это в совокупности делает эксперименты с алгоритмами на Python более удобными, чем на Java. Тем не менее, перенос экспериментов на чистую Java - перспективное направление дальнейшей работы
\item Выбранное решение - экспорт параметров из Python-модели в промежуточный формат (а именно, JSON) и дальнейший импорт в Java-модель. Так как распространённые алгоритмы, в частности, нейронные сети, в самых различных реализациях работают практически идентично, значительного различия между Python и Java при выполнении классификации не возникает
\end{itemize}

\subsection{Архитектура Android-приложений}

Java-код разбит на несколько подпроектов:

\begin{itemize}
\item javacommon - библиотека, реализующую бизнес-логику распознавания. Включает в себя следующие пакеты:
  \begin{itemize}
  \item events - классы, описывающие временные ряды (далее - \defn{логов}) и имеющиеся в них события, а также интерфейсы для источников этих событий
  \item transfer - интерфейсы для систем обмена сообщениями между часами и смартфоном, а также роли участников этого обмена (отправляющая/принимающая сторона)
  \item data\_channel - интерфейс для канала отправки данных в офлайн-систему
  \item utils - код для предобработки логов и их сжатия
  \end{itemize}
Выделение этого кода в отдельный подпроект позволяет абстрагироваться от особенностей Android и тестировать бизнес-логику независимо от интерфейса
\item androidcommon - общий для смартфона и часов код; зависит от javacommon. Наиболее важным модулем является диспетчер сообщений, который инкапсулирует процедуру обмена сообщениями посредством Android Message API.
\item mobile, wear - приложения для смартфона и часов соответственно; зависят от javacommon и androidcommon. Включают в себя экраны интерфейса и код для запуска обмена данными
\end{itemize}

Для управлениями зависимостями, тестами и сборкой используется Gradle.

Для более подробного описания архитектуры рассмотрим, как взаимодействуют компоненты при обмене данными.

\begin{enumerate}
\item На часах и смартфоне запускаются приложения.
  \begin{itemize}
  \item  Для часов используется базовый экран WearControlScreen, который используется исключительно для отображения полученных и отправленных сообщений
  \item На смартфоне используются два экрана, соответствующие двум режимам работы: LogTransferScreen для сбора данных и отправки в офлайн-систему и LogReceiverScreen, который управляет распознаванием движений. Оба этих экрана имеют одни и те же элементы управления - кнопки "начать сбор данных" и "прекратить сбор данных"
  \end{itemize}
\item По нажатию на кнопку "Начать сбор данных" смартфон отправляет соответствующее сообщение через диспетчер сообщений. Так как ему известен только интерфейс этого диспетчера, конкретные реализации можно свободно заменять. В ходе работы было разработано два диспетчера - использующие Wearable Data Layer API и Message API. Было установлено, что Wearable Data Layer не подходит к этой задаче, так как предполагает, что обмен данными заключается в синхронизации отдельных объектов. Обмен сообщениями плохо ложится в эту концепцию, поэтому для работы был выбран Message API.
\item На стороне часов WearControlScreen при инициализации создаёт объект WearPeer, реализующий интерфейс MessageReceiver. Этот интерфейс предназначен для объектов, которые могут получать сообщения от диспетчера и выполнять в ответ какие-либо действия. Этот интерфейс независим от механизма передачи сообщений, что значительно упрощает его тестирование и реализацию. Конкретно WearPeer отвечает за начало и окончание записи логов. В качестве аргументов конструктора он принимает диспетчер сообщений, источник логов и объект для обратного вызова (callback). Необходимость в callback обусловлена тем, что обработка сообщений происходит асинхронно основному потоку, и необходимо иметь возможность уведомлять другие компоненты системы о результатах обработки. В качестве callback передаются экраны приложений, которые отображают результаты обработки. Как и диспетчер сообщений, callback передаётся как объект соответствующего интерфейса
\item При получении сообщения "начать сбор данных" WearPeer проверяет своё текущее состояние (ожидание/запись). Если он находится в ожидании, то вызывается метод источника логов, который возвращает объект сессии логирования. Сессия используется для разделения логов, созданных в разное время. Она также представлена интерфейсом LoggingSession; конкретная реализация зависит от источника логов. Этот источник задаётся при создании WearPeer; в качестве параметра он принимает тип сенсора. Используемая реализация возвращает объект сессии, который ожидает вызова со стороны ОС для приёма новых данных сенсора. Android не позволяет получать данные с точно заданной частотой - можно только указать её приблизительно. На практике частота поступления событий составляет примерно 70 Гц
\item При получении новых данных сессия записывает в объект ActionLog новое событие (ActionEvent). Событие состоит из временной метки (представлена 8-байтным типом long) и массива значений сенсора. ActionLog - это вспомогательный объект, содержащий список событий и используемый, в частности, для контроля размерности добавляемых событий и максимальной длины лога.
\item При получении сообщения "прекратить сбор данных" WearPeer вызывает на сессии метод stopAndRetrieve, который останавливает получение данных от ОС и возвращает ActionLog с записанными событиями
\item Полученные события обрабатываются объектом EventsLogCompressor. В результате обработки возвращается байтовый массив, каждые 20 байт которого обозначают отдельное событие: 8 байт на временную метку, 3 значения типа float (по 4 байта) для значений сенсоров. Google рекомендует отправлять сообщения размером не больше 100 Кбайт; для упрощения расчётов примем максимальный размер равным 100000 байт. Тогда в одном сообщении можно отправить не более 5000 событий. С учётом частоты поступления событий, в сообщении можно передать данные о событиях приблизительно за последние 70 секунд
\item Сообщение типа "лог" через диспетчер отправляется смартфону
\item На стороне смартфона также используется объект с интерфейсом MessageReceiver. В зависимости от режима, это либо LogProcessingPeer (управляет классификацией логов), либо LogTransferPeer (отправляет данные в офлайн-систему). LogProcessingPeer с помощью EventsLogCompressor преобразует байтовый массив в лог. Затем он вызывает объект интерфейса LogClassifier, чтобы произвести классификацию. LogTransferPeer же отправляет полученный байтовый массив для дальнейшей офлайн-обработки
\item LogClassifier представляет собой интерфейс к классификатору логов. Он инкапсулирует как алгоритм выделения признаков, так и непосредственно классификатор. Основной метод - classify, который принимает лог и возвращает один из объектов перечисления (enum), соответствующий той или иной активности
\end{enumerate}

\missing{улучшить структуру раздела}

\missing{отдельный шрифт для кода}

\subsection{Архитектура системы обучения моделей}

Задача системы обучения - проведение экспериментов с разными алгоритмами выделения признаков и классификации, а также подготовка файлов с параметрами обученных моделей.

Общий ход выполнения эксперимента задан в модуле \code{experiments.run\_experiment}. 

\begin{enumerate}
\item В зависимости от переданных аргументов из всего набора доступных алгоритмов выбирается некоторое подмножество. Это обусловлено тем, что при локальной отладке запуск всех алгоритмов занимает слишком большое время, и удобно выбрать только один из них. Кроме того, на сервере Travis CI (см. раздел про непрерывную интеграцию) установлен лимит времени в 50 минут на выполнение всех операций. При использовании кросс-валидации и переборе всех гиперпараметров время выполнения значительно превышает этот лимит, поэтому необходимо указывать подмножество наиболее важных алгоритмов, которые будут запускаться на сервере
\item Алгоритмы выделения признаков и классификации попарно комбинируются во всех различных сочетаниях. Таким образом, происходит исследование совместного функционирования этих алгоритмов
\item Производится загрузка и предобработка логов (см. далее)
\item Для каждой из ранее выбранной пар алгоритмов выполняется эксперимент с загруженными логами. В качестве результата возвращается матрица ошибок, рассчитанная при оптимальных значениях гиперпараметров на тестовой выборке, и оптимальные значения параметров модели и гиперпараметров алгоритма
\item Результаты сортируются по убыванию F-оценки (она же используется при кросс-валидации); таким образом, наглядно отображаются наиболее перспективные алгоритмы
\item Параметры классификаторов экспортируются в файлы и загружаются в хранилище Dropbox
% \missing{доработать алгоритм выгрузки параметров!}
\end{enumerate}

Рассмотрим подробнее этап загрузки логов:

\begin{enumerate}
\setcounter{enumi}{0}
\item Предварительный этап - преобразование бинарных файлов в текстовый формат. Как было указано ранее, в Android-приложении логи преобразуются в бинарный формат и отправляются в хранилище Dropbox. Чтобы преобразовать их в вид, понятный пакету pandas, наиболее целесообразно воспользоваться тем же модулем, которым было произведено сжатие. Для этого был написан небольшой скрипт на Groovy, который импортирует библиотеку javacommon и с её помощью сохраняет логи в текстовые файлы. Сохранение производится в CSV-файл с именем, в котором указан целевой класс. Этот этап выполняется вручную для обеспечения тщательного контроля данных, которые используются в экспериментах.
\item Модулю загрузки логов передаётся путь до каталога с файлами. Далее все найденные файлы используются для создания объектов \code{DataFrame} пакета pandas. Эти объекты представляют собой обёртку над многомерными массивами с возможностью индексации по временным меткам и другими дополнительными возможностями
\item Так как временные метки в "сырых" логах распределены неравномерно, то для корректного выделения признаков данные необходимо преобразовать так, чтобы временные метки были распределены с одинаковым шагом. Это выполняется усреднением точек, попадающих в один временной интервал. В результате экспериментов было установлено, что интервал в 100 мс даёт достаточно данных для дальнейшей обработки
\item Наконец, логи разбиваются на промежутки определённой длины. Это обусловлено тем, что при обработке в реальном времени классификатор должен реагировать достаточно быстро, и из-за слишком длинных логи отклик приложения будет долгим. Было решено использовать логи длиной в 1 секунду. Кроме того, удаляется по секунде в начале и в конце исходных логов, так как в это время обычно происходит подготовка к упражнению
\end{enumerate}

Ход эксперимента заключается в следующем:

\begin{enumerate}
\item Создаются 5 подвыборок для выполнения кросс-валидации, а также выделяется тестовая выборка
\item По указанным алгоритмам выделения признаков и классификации создаётся объект \code{Pipeline} пакета sklearn. Это вспомогательный объект, имеющий тот же интерфейс, что и обычные классификаторы. Здесь отметим, что классификаторы в sklearn имеют метод \code{fit}, который производит оптимизацию параметров модели. \code{Pipeline} при вызове этого метода оптимизирует все включённые в него модели, что упрощает дальнейшее выполнение эксперимента
\item Полученный \code{Pipeline} передаётся в конструктор объекту \code{GridSearchCV}. При вызове метода \code{fit} на нём производится оптимизация гиперпараметров включённых в него моделей. В качестве метрики качества используется F-мера с macro-усреднением метрик отдельных классов
\item Благодаря последовательной инкапсуляции алгоритмов друг в друга процедура обучения в эксперименте сводится к единственному вызову \code{fit}
\item Оценивается качество работы алгоритма на тестовой выборке
\item В качестве результата эксперимента возвращаются матрица ошибок и набор оптимальных параметров
\end{enumerate}

Базовым для классов-выделителей признаков является интерфейс \code{BaseEstimator}, предоставляемый sklearn. Для объединения всех реализуемых алгоритмов был введён класс \code{LogFeatureExtractor}, реализующий этот интерфейс. Объект этого класса вызывает для каждого полученного лога абстрактный метод \code{extract\_item\_features}, который реализуется конкретными подклассами для выделения признаков из лога. 

Все используемые в экспериментах классификаторы уже реализованы в sklearn, поэтому для их использования дополнительный код писать не пришлось. 

