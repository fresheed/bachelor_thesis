
\section{Процесс тестирования и непрерывной интеграции}

В обеих частях системы присутствуют модульные тесты, покрывающие наиболее важные части кода. 

Тесты на Python реализованы на стандартном фреймворке unittest. БОльшая часть Python-кода, не содержащая бизнес-логики и вызывающая стандартные для sklearn процедуры, не покрыта тестами. Наибольшее покрытие обеспечено для модуля classification.logs\_loading, обеспечивающего загрузку и предобработку логов.

Тесты для Java-кода реализованы на фреймворке JUnit 4. Тестами покрыта бОльшая часть библиотеки javacommon, т.к. для этого не требуется взаимодействие с Android-кодом. Наибольшее внимание уделено классам, реализующим интерфейс MessageReceiver, и классам обработки логов.

Так как код, не покрытый тестами, нуждается хотя бы в минимальном контроле, необходим способ его автоматической компиляции и запуска. Кроме того, необходимо быть уверенным в том, что при локальной разработке не были добавлены неучтённые зависимости, и стандартный процесс сборки не был нарушен. 

Для решения этой задачи в проекты были добавлены конфигурации Travis CI - сервера для обеспечения непрерывной интеграции (CI). В процессе выполнения процедур на CI-сервере для обеих частей системы загружаются зависимости, выполняются модульные тесты, производится сборка. Кроме того, для системы обучения моделей выполняются все доступные эксперименты (в рамках доступного времени - 50 минут). 

Использование Travis CI позволяет, во-первых, гарантировать изоляцию окружения сборки от локального окружения, во-вторых, обеспечить запуск сборки по каждому коммиту в систему контроля версий, в-третьих, избежать траты ресурсов локального компьютера на проведение сборок. 