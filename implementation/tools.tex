
\section{Выбор средств и окружения для реализации}

Для обучения моделей в офлайн-режиме используется фреймворк scikit-learn версии 0.18.1 для языка Python версии 3.4.3. Выбор обусловлен наличием большого количества реализованных алгоритмов классификации и удобным API, которое позволяет добавлять новые модули и встраивать их в стандартный процесс обучения. 

Для работы с временными рядами используется библиотека pandas версии 0.19.2.

Для выделения признаков используются библиотеки:

\begin{itemize}
\item hmmlearn версии 0.2.0 - для работы со скрытыми марковскими моделями
\item statsmodels версии 0.8.0 - для построения моделей авторегрессии
\item pywavelets версии 0.5.2 - для выполнения вейвлет-преобразований
\item ucrdtw версии 0.201 - для вычисления расстояния DTW
\end{itemize}

Для выполнения распознавания в реальном времени используются библиотеки:

\begin{itemize}
\item Java Statistical Analysis Tool - для выполнения классификации
\item TarsosDSP - для выполнения преобразования Фурье
\end{itemize}

