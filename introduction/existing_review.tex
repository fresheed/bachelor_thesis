%\section{Обзор существующих решений и постановка задачи}
\chapter{Обзор существующих решений и постановка задачи}

В настоящий момент производители электроники предлагают большое количество устройств, которые позволяют отслеживать активность. Во-первых, это специализированные фитнес-трекеры, которые используются исключительно для определения активности (производители: Garmin, Fitbit, Withings и многие другие). Во-вторых, это умные часы, которые сочетают свойства фитнес-трекера и портативного компьютера, синхронизирующегося со смартфоном (их производят такие компании, как Sony, Asus, Motorola). 

Был произведён обзор существующих средств распознавания движений. Приведём характеристики некоторых рассмотренных устройств: \missing{переформулировать буллеты}

% https://www.fitbit.com/smarttrack
% https://us.community.samsung.com/t5/Wearable-Tech/Gear-fit2-not-auto-tracking-exercise/m-p/9556#U9556
% https://actofit.com/
% http://www.bestfitnesstrackerreviews.com/wearables-for-auto-detecting-many-exercises.html
\begin{itemize}
\item В новых моделях трекеров Fitbit присутствует опция SmartTrack. С её помощью выполняется распознавание заранее заданных типов активности (ходьба, бег, аэробные нагрузки, работа на эллиптическом тренажёре, катание на велосипеде, плавание и общая категория "спорт"). При этом для распознавания активности требуется, чтобы она длилась не менее 10 минут. Отметим, что это ограничение встречается достаточно часто во многих моделях трекеров (например, Samsung Gear)
\item ActoFit - ориентирован на распознавание упражнений в спортзале; автоматически производит распознавание как типа упражнения, так и числа повторов; по окончании тренировки показывает подробную статистику. В данный момент доступен только для предзаказа (так же, как и трекер Atlas)
\item Moov предлагает использовать как браслет, так и нагрудную повязку. Благодаря этому подробная информация о выполняемом упражнении выдаётся в реальном времени. Трекеры поддерживают распознавание ходьбы, бега, плавания, кардиотренировок и упражнений с собственным весом
\end{itemize}

Также рассмотрим решения на основе умных часов:
\begin{itemize}
\item Многие модели, основанные на Android Wear 1.0, поддерживают распознавание активности с помощью приложения Google Fit. При этом пользователь должен самостоятельно выбрать тип активности, и после этого начинается отслеживание его выполнения \missing{проверить}. Также для этих моделей распространены приложения, которые используют уже обработанные в Fit данные для уведомления пользователей, например, о слишком долгом отсутствии движения
\item Android Wear 2.0 добавляет возможность автоматического распознавания типов упражнений и повторов; тем не менее, эта возможность поддерживается не на всех устройствах (среди поддерживающих - LG Sport Watch)
\item Приложение TrackMyFitness производит самостоятельный анализ активности. Возможно добавление новых типов упражнений \missing{опыт использования?}
\end{itemize}

Подведём итог. Основная масса трекеров и приложений предоставляет возможность отслеживания длительной активности одного из предопределённых типов. Средства, предоставляющие расширенные возможности (анализ в реальном времени, более подробная информация, добавление новых типов активности) либо ещё недостаточно распространены среди рядовых пользователей, либо требуют дополнительных устройств помимо браслета. 

%Исходя из этого, в данной работе реализуется приложение, которое должно запо
%Исходя из выявленных недостатков существующих средств, были поставлены следующие требования

Проект, реализуемый в данной работе, ориентирован на реализацию недостающих возможностей существующих средств, в первую очередь - отслеживание активности в реальном времени\missing{дополнить?}. Источником данных являются умные часы Sony SmartWatch 3 (на основе Android Wear 1.0), анализ данных происходит на смартфоне с ОС Android 5. Выбор этих средств обусловлен:
\begin{itemize}
\item ценовой доступностью
\item возможностью доступа к "сырым" данным датчиков стандартными средствами Wear (такая возможность отсутствует во всех фитнес-трекерах)
\item наличием протоколов обмена данными между этими устройствами, что обусловлено принадлежностью к одному семейству ОС
\end{itemize}

