\intro

В 2013 году ВОЗ опубликовала статистический профиль России - документ, в котором описываются такие базовые характеристики национального здравоохранения, как средняя продолжительность жизни, затраты на медицину и т.д\cite{who_russia_profile}. Среди прочих в нём указаны важнейшие факторы риска. Самые значимые из них - сердечно-сосудистые заболевания и ожирение. Самые распространённые причины смерти - ишемическая болезнь сердца и инсульт. 

По данным той же ВОЗ\cite{who_cardiovascular}, главным средством профилактики сердечно-сосудистых заболеваний является ведение здорового образа жизни - физическая активность, правильное питание, а также отказ от курения. Поэтому все средства, которые позволяют следить за своим образом жизни, имеют большое значение в поддержании здоровья населения.

В последнее время растёт интерес к т.н. \defn{носимым устройствам} - электронным сенсорам, которые постоянно носятся на одежде или непосредственно на теле человека. Среди их возможных функций - отслеживание движений, местоположения, некоторых параметров жизнедеятельности (например, текущий пульс). Эти устройства имеют несколько отличительных особенностей, которые делают их полезными в задаче контроля за образом жизни:
\begin{itemize}
\item постоянное присутствие на теле человека позволяет получать непрерывный и актуальный поток информации о состоянии здоровья
\item обработка информации может производиться удалённо - на смартфонах или в дата-центрах, что позволяет не ограничиваться вычислительной мощностью отдельного устройства
\item доступная цена делает возможным их применение как специалистами в клиниках, так и рядовыми пользователями
\end{itemize}

Задача классификации активности по данным датчиков достаточно трудно формализуется. Это обусловлено, в первую очередь, различиями в характере движений у разных пользователей. Поэтому в данной области активно используются методы машинного обучения, с помощью которых система обучается классификации наборов данных без необходимости явного и ручного задания признаков. В последнее время эти методы развиваются крайне быстро, что обусловлено как ростом вычислительной мощности (в первую очередь - развитие вычислений на GPU), так и появлению больших массивов данных, на которых можно проводить эффективное обучение.

Одним из классов носимых устройств являются \defn{фитнес-трекеры}. Они представляют собой мини-компьютеры, носимые на запястье и используемые для определения физической активности. Источниками данных в них служат в основном многоосевые акселерометры и гироскопы; некоторые устройства также измеряют сердечный пульс и местоположение пользователя. 

Носимые устройства используются как для решения узких задач, так и для более общего мониторинга активности. Среди решаемых задач можно отметить детектирование падений\cite{applications_falls}, приступов болезни Паркинсона\cite{applications_parkinson} и приступов эпилепсии\cite{applications_epileptic}. Определение же текущей активности в основном используется для контроля тренировок в спортзале, а также распознавания ходьбы. Интересным применением таких сенсоров являются проекты дополненной реальности: например, с помощью закреплённого на спине сенсора и камеры выполняется интерактивная тренировка с целью укрепления мышц спины, при этом система даёт подсказки в реальном времени\cite{applications_vr}. 

Подводя итог вышесказанному, можно с уверенностью сказать, что носимые устройства играют всё большую роль в поддержании здорового образа жизни.
