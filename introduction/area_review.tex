\chapter{Обзор предметной области}

\section{Постановка задачи}

Цель работы: реализация системы распознавания двигательной активности, работающей на смартфоне и в реальном времени обрабатывающей данные с носимых устройств.

Решаемые задачи:
\begin{itemize}
\item Обзор предметной области, поиск аналогов и существующих подходов к решению задачи;
\item Сбор массива данных для обучения и тестирования;
\item Реализация отдельных модулей системы - для сбора данных, обучения классификаторов и обработки данных в реальном времени;
\item Эксперименты с различными алгоритмами, оптимизация их параметров и выбор наиболее подходящих.
\end{itemize}

Объект исследования - система распознавания движений.

Предмет исследования - выбор алгоритмов для решения задачи классификации двигательной активности по данным с носимых устройств. 

\section{Существующие решения}

В настоящий момент производители электроники предлагают большое количество устройств, которые позволяют отслеживать активность. Во-первых, это специализированные фитнес-трекеры, которые используются исключительно для определения активности (производители: Garmin, Fitbit, Withings и многие другие). Во-вторых, это умные часы, которые сочетают свойства фитнес-трекера и портативного компьютера, синхронизирующегося со смартфоном (их производят такие компании, как Sony, Asus, Motorola). 

Был произведён обзор существующих средств распознавания движений. В результате было выделено несколько классов решений. Рассмотрим сначала решения для специализированных фитнес-трекеров:
% https://www.fitbit.com/smarttrack
% https://us.community.samsung.com/t5/Wearable-Tech/Gear-fit2-not-auto-tracking-exercise/m-p/9556#U9556
% https://actofit.com/
% http://www.bestfitnesstrackerreviews.com/wearables-for-auto-detecting-many-exercises.html
\begin{itemize}
\item Наибольшая группа - устройства, распознающие один из заранее заданных типов активности. При этом характерные движения должны совершаться на протяжении 10 и более минут. Представители этой группы - новые модели трекеров Fitbit, имеющие опцию SmartTrack. С её помощью выполняется распознавание заранее заданных типов активности (ходьба, бег, аэробные нагрузки, работа на эллиптическом тренажёре, катание на велосипеде, плавание и общая категория "спорт"). Аналогичные возможности имеет трекер Samsung Gear
\item Трекеры, состоящие из нескольких носимых сенсоров, например, Moov. Он предлагает использовать как браслет, так и нагрудную повязку. Благодаря этому подробная информация о выполняемом упражнении выдаётся в реальном времени. Moov поддерживает распознавание ходьбы, бега, плавания, кардиотренировок и упражнений с собственным весом
\item Трекеры "нового поколения" - по утверждениям разработчиков, автоматически производит распознавание как типа упражнения, так и числа повторов; по окончании тренировки показывает подробную статистику. Такие устройства пока что не получили достаточного распространения: ActoFit (ориентирован на распознавание упражнений в спортзале) только начал поступать первым покупателям, Atlas в данный момент доступен только для предзаказа
\end{itemize}

Также рассмотрим решения на основе умных часов:
\begin{itemize}
\item Многие модели, основанные на Android Wear 1.0, поддерживают распознавание активности с помощью приложения Google Fit. При этом пользователь должен самостоятельно выбрать тип активности, и после этого начинается отслеживание его выполнения. Некоторые длительные типы активности (ходьба, езда на велосипеде) отслеживаются автоматически. Также для этих моделей распространены приложения, которые используют уже обработанные в Fit данные для уведомления пользователей, например, о слишком долгом отсутствии движения
\item Android Wear 2.0 добавляет возможность автоматического распознавания типов упражнений и повторов; тем не менее, эта возможность поддерживается не на всех устройствах (среди поддерживающих - LG Sport Watch)
\item Приложение TrackMyFitness производит самостоятельный анализ активности. Пользователь может добавлять новые типы упражнений. Данное приложение было установлено и изучено: распознавание работает с достаточно высокой точностью, однако встречаются и ошибки
\end{itemize}

Подведём итог. Основная масса трекеров и приложений предоставляет возможность отслеживания длительной активности одного из предопределённых типов. Средства, предоставляющие расширенные возможности (анализ в реальном времени, более подробная информация, добавление новых типов активности) либо ещё недостаточно распространены среди рядовых пользователей, либо требуют дополнительных устройств помимо браслета. 

\section{Существующие подходы к решению задачи}

Существует большое количество исследований, посвящённых распознаванию движений по данным с сенсоров. 
\begin{itemize}
\item Lester et al.\cite{review_lester} в качестве носимого устройства использовал специально сконструированный сенсор на ремне. В качестве исходных данных использовались сигналы с акселерометра, микрофона и барометра. Изначально выделялось около 600 признаков, затем с помощью AdaBoost их число снижалось до 50. В основном в их число входили спектральные признаки. В качестве классификатора использовалась скрытая марковская модель.
\item Anguita et al.\cite{review_anguita} использовал акселерометр и гироскоп смартфона, закреплённого на поясе. Из исходных данных суммарно извлекалось 17 признаков временной и частотной областей. Для классификации использовался SVM
\item Garcia-Ceja и Brena\cite{review_garcia} извлекали всего 4 признака из данных каждой оси акселерометра: среднее, стандартное отклонение, энергия, корреляция с другими осями. Авторы рассматривали как отдельные классификаторы (SVM, KNN, наивный байесовский классификатор, решающие деревья), так и их композации. В результате было установлено, что наилучший результат получается при голосовании большинством из всех классификаторов
\item Mitchell, Monaghan и O'Connor\cite{review_mitchell} в качестве источника данных использовали акселерометр смартфона, закреплённого на спине. Для извлечения признаков использовалось дискретное вейвлет-преобразование. Среди классификаторов наилучший результат показал наивный байесовский классификатор
\item Mannini et al.\cite{review_mannini} сравнивали эффективность сенсоров, закреплённых на колене и запястье и показали, что выбор его оптимального положения сильно зависит от типа активности. Для выделения признаков также использовалось вейвлет-преобразование, для классификации - SVM
\item Lee и Cho\cite{review_lee} обрабатывали данные с акселерометра смартфона с помощью комбинации двух скрытых марковских моделей: одна извлекала отдельные действия, другая по последовательности действий определяла тип активности
\end{itemize}

Видно, что нельзя выделить один метод, который был оптимален во всех случаях. Авторы экспериментировали с целыми наборами алгоритмов, находя подходящий именно к их задаче, что и будет произведено в данной работе. Тем не менее, можно выделить основные подходы к решению задачи:

\begin{itemize}
\item В основном использовались данные с акселерометра, другие сенсоры использовались реже
\item Наиболее часто используемые классификаторы - SVM, скрытые марковские модели, наивный байесовский классификатор
\item Распознавание производилась с небольшим количеством признаков; при наличии большого числа признаков выбирались наилучшие
\end{itemize}


