\chapter{Актуальность задачи распознавания движений}

\missing{переформулировать начало}
В 2013 году ВОЗ опубликовала статистический профиль России. Это документ, в котором описываются такие базовые характеристики национального здравоохранения, как средняя продолжительность жизни, затраты на медицину и т.д. Среди прочих в нём указаны важнейшие факторы риска. Самые значимые из них - сердечно-сосудистые заболевания и ожирение. Самые распространённые причины смерти - ишемическая болезнь сердца и инсульт. 
\missing{ссылку} % http://www.who.int/publications/list/PocketGL_Russian.pdf?ua=1
% http://www.who.int/gho/countries/rus.pdf?ua=1

По данным той же ВОЗ\missing{ссылку}, главным средством профилактики сердечно-сосудистых заболеваний является ведение здорового образа жизни - физическая активность, правильное питание, а также отказ от курения. Поэтому все средства, которые позволяют следить за своим образом жизни, имеют большое значение в поддержании здоровья населения.

В последнее время растёт интерес к т.н. \defn{носимым устройствам} - электронным сенсорам, которые постоянно носятся на одежде или непосредственно на теле человека. Среди их возможных функций - отслеживание движений, местоположения, некоторых параметров жизнедеятельности (например, текущий пульс). Эти устройства имеют несколько достоинств, которые делают их важными средствами поддержания здорового образа жизни\missing{перефразировать}:
\begin{itemize}
\item постоянное присутствие на теле человека позволяет получать непрерывный и актуальный поток информации о состоянии здоровья
\item обработка информации может производиться удалённо - на смартфонах или в дата-центрах, что позволяет не ограничиваться вычислительной мощностью отдельного устройства
\item доступная цена делает возможным их применение как специалистами в клиниках, так и рядовыми пользователями
\end{itemize}

% http://download.springer.com/static/pdf/307/art%253A10.1186%252F1743-0003-9-21.pdf?originUrl=http%3A%2F%2Fjneuroengrehab.biomedcentral.com%2Farticle%2F10.1186%2F1743-0003-9-21&token2=exp=1496651210~acl=%2Fstatic%2Fpdf%2F307%2Fart%25253A10.1186%25252F1743-0003-9-21.pdf*~hmac=a427e14ee16c8ee1bd0a77a2c548da9cba600f995fd51176ffcf84f38b2a166c
Такие устройства уже используются как для решения узких задач, так и для более общего мониторинга активности. Среди решаемых задач можно отметить детектирование падений \missing{ссылка}, приступов болезни Паркинсона \missing{ссылка}\missing{перефразировать} и приступов эпилепсии\missing{ссылка}. Определение же текущей активности позволяет собирать статистику об образе жизни и на её основе строить рекомендации по его улучшению. Интересным применением таких сенсоров являются проекты дополненной реальности: например, с помощью закреплённого на спине сенсора и камеры выполняется интерактивная тренировка с целью укрепления мышц спины, при этом система даёт подсказки в реальном времени \missing{ссылка}. 

Одним из классов носимых устройств являются \defn{фитнес-трекеры}. Они представляют собой мини-компьютер, носимый на запястье и используемый для определения физической активности. Источниками данных в них служат в основном многоосевые акселерометры и гироскопы; некоторые устройства также измеряют сердечный пульс и местоположение пользователя. 


умные часы


google fit


МО