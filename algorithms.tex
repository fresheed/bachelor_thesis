
\chapter{Обзор алгоритмов решения задачи}

В наиболее абстрактном виде задача описывается следующим образом. 

Временные ряды представлены конечными последовательностями значений: $S=s_1,s_2...s_n$, где $s_i$ - вектор значений $\left\{x_i,y_i,z_i\right\}$. Дан конечный набор классов $C=c_1,c_2...c_m$. Известно, что каждый ряд принадлежит одному и только одному классу. 

Для некоторого множества $L$ рядов известна принадлежность ряда к классу - такое множество назовём \defn{обучающей выборкой}. Также имеется множество $T$ рядов, принадлежность которых к классам неизвестна - это \defn{тестовая выборка}.

Необходимо по данной обучающей выборке классифицировать элементы тестовой выборки с максимальной точностью.


\section{Алгоритмы, основанные на метриках расстояния}

Существует класс алгоритмов, основанных на вычислении расстояния (\defn{метрики}) между классифицируемыми объектами. 

\subsection{Алгоритм динамической трансформации временной шкалы (DTW)}

Основная идея алгоритма DTW - минимизация эффекта сдвигов и искажений сигналов во времени путём трансформации одного временного ряда в другой. Это позволяет находить сигналы схожих форм с разными фазами\cite{dtw_review}. 

Пусть даны два временных ряда $X=x_1,x_2...x_N$ и $Y=y_1,y_2...y_M$. Элементы рядов $x_i,y_j$ принадлежат одному пространству \missing{уточнить} $\Phi$, для которого введена метрика $\Phi \times \Phi \rightarrow R$

\missing{дополнить}
Время работы алгоритма - $O(M*N)$
есть оптимизации



\missing{confmat ?}. 

\missing{другие метрики}



